\documentclass{article}
\usepackage{multicol}
\usepackage{float}
\usepackage[a4paper,margin=1.2in]{geometry}

\title{The latex-formatter CTAN package \\ v0.5.6}
\author{William G.\ Underwood}

\begin{document}

\maketitle

\section{Overview}

The latex-formatter CTAN package provides the tex-fmt command-line
tool for formatting LaTeX source files.

\renewcommand{\arraystretch}{1.8}

\begin{itemize}
  \item Extremely fast run-time performance
  \item Minimal configuration required
  \item Command line interface
  \item Handles LaTeX file types \texttt{.tex},
    \texttt{.bib}, \texttt{.cls}, and \texttt{.sty}
  \item Written entirely in safe Rust
\end{itemize}

\begin{table}[H]
  \begin{center}
    \begin{tabular}{|c|c|}
      \hline
      {\large Before \texttt{tex-fmt file.tex}}
      &
      {\large After \texttt{tex-fmt file.tex}}
      \\
      \hline
      \begin{minipage}{5cm}
        \vspace*{2mm}
        \begin{verbatim}
\documentclass{article}

\begin{document}

\begin{itemize}
\item Lists with items
over multiple lines
\end{itemize}

\begin{equation}
E = m c^2
\end{equation}

\end{document}
        \end{verbatim}
      \end{minipage}
      &
      \begin{minipage}{5cm}
        \vspace*{2mm}
        \begin{verbatim}
\documentclass{article}

\begin{document}

\begin{itemize}
  \item Lists with items
    over multiple lines
\end{itemize}

\begin{equation}
  E = m c^2
\end{equation}

\end{document}
        \end{verbatim}
      \end{minipage}
      \\
      \hline
    \end{tabular}
  \end{center}
\end{table}

\pagebreak

\section{Usage}

The most commonly used options are given below.
For a full list, see Section~\ref{sec:options}.

\begin{verbatim}
tex-fmt file.tex                   # format file.tex and overwrite
tex-fmt --check file.tex           # check if file.tex is correctly formatted
tex-fmt --print file.tex           # format file.tex and print to stdout
tex-fmt --recursive                # recursively format files in
                                   # current directory
tex-fmt --recursive dir/           # recursively format files in dir
tex-fmt --fail-on-change file.tex  # format file.tex and return exit-code 1
                                   # if overwritten
tex-fmt --nowrap file.tex          # do not wrap long lines
tex-fmt --stdin                    # read from stdin and print to stdout
tex-fmt --help                     # view help information
\end{verbatim}

\subsection{Configuration}

Options can also be set using a configuration file, which
will be read from the following locations, in order of decreasing priority.

\begin{itemize}
  \item A named config file passed as \texttt{tex-fmt --config <PATH>}
  \item A file named \texttt{tex-fmt.toml} in the current working directory
  \item A file named \texttt{tex-fmt.toml} in the root directory of
    the current git repository
  \item A file named \texttt{tex-fmt.toml} in a subdirectory titled
    \texttt{tex-fmt/}
    in the user's configuration directory at
    %
    \begin{itemize}
      \item Linux: \texttt{\textasciitilde/.config/tex-fmt/tex-fmt.toml}
      \item macOS: \texttt{/Users/<user>/Library/Application
        Support/tex-fmt/tex-fmt.toml}
      \item Windows:
        \texttt{C:\textbackslash
          Users\textbackslash<user>\textbackslash AppData\textbackslash
        Roaming\textbackslash tex-fmt\textbackslash tex-fmt.toml}
    \end{itemize}

\end{itemize}

Arguments passed on the command line will always override those
specified in configuration files.
To ignore all config files, use the \texttt{--noconfig} flag.
An example configuration file is as follows.

\begin{verbatim}
# tex-fmt.toml
check = false
print = false
wrap = true
wraplen = 80
tabsize = 2
tabchar = "space"
stdin = false
verbosity = "warn"
lists = []
no-indent-envs = []
\end{verbatim}

\pagebreak

\subsection{Disabling the formatter}

Ending a source line with \texttt{\% tex-fmt: skip}
disables formatting for that line.
To disable the formatter for a block, use \texttt{\% tex-fmt: off}
and \texttt{\% tex-fmt: on}.
Verbatim environments including
\texttt{verbatim},
\texttt{Verbatim},
\texttt{lstlisting}
and \texttt{minted} are automatically skipped.

\begin{verbatim}
\documentclass{article}
\begin{document}
    This line is skipped % tex-fmt: skip
% tex-fmt: off
  These lines are also
    not formatted or wrapped
% tex-fmt: on
\end{document}
\end{verbatim}

\subsection{Ignoring files in recursive mode}

Recursive searches with \texttt{--recursive} or \texttt{-r} will
ignore patterns in \texttt{.gitignore} and \texttt{.ignore} files,
following git conventions.

\subsection{Shell completion}

Shell completion scripts can be generated at run-time using the
\texttt{--completion <SHELL>} flag.

\subsection{Man page}

A man page can be generated at run-time using the
\texttt{--man} flag.

\section{Limitations}

\begin{itemize}
  \item Semantic parsing of LaTeX code not conducted
  \item No linting or correction of syntax errors
  \item Compliance with existing formatting guidelines not guaranteed
  \item No spelling or grammar checking
\end{itemize}

\pagebreak

\section{Options}
\label{sec:options}

\subsection{Command line options}

The following arguments can be passed on the command line.

\renewcommand{\arraystretch}{1.2}

\begin{table}[H]
  \begin{center}
    \begin{tabular}{|l|l|l|l|}
      \hline
      \multicolumn{1}{|c|}{Option}
      & \multicolumn{1}{|c|}{Alias}
      & \multicolumn{1}{|c|}{Default}
      & \multicolumn{1}{|c|}{Description} \\
      \hline
      \texttt{--check}              & \texttt{-c}  &         & Check
      formatting, do not modify files \\
      \texttt{--print}              & \texttt{-p}  &         & Print
      to stdout, do not modify files \\
      \texttt{--fail-on-change}     & \texttt{-f}  &         & Fail
      if files are modified \\
      \texttt{--recursive}          & \texttt{-r}  &         &
      Recursively search for files to format \\
      \texttt{--nowrap}             & \texttt{-n}  &         & Do not
      wrap long lines \\
      \texttt{--wraplen <N>}        & \texttt{-l}  & \texttt{80}    &
      Line length for wrapping \\
      \texttt{--tabsize <N>}        & \texttt{-t}  & \texttt{2}     &
      Number of characters to use as tab size \\
      \texttt{--usetabs}            &       &         & Use tabs
      instead of spaces for indentation \\
      \texttt{--stdin}              & \texttt{-s}  &         &
      Process stdin as a single file, output to stdout \\
      \texttt{--config <PATH>}      &       &         & Path to config file \\
      \texttt{--noconfig}           &       &         & Do not read
      any config file \\
      \texttt{--verbose}            & \texttt{-v}  &         & Show
      info messages \\
      \texttt{--quiet}              & \texttt{-q}  &         & Hide
      warning messages \\
      \texttt{--trace}              &       &         & Show trace messages \\
      \texttt{--completion <SHELL>} &       &         & Generate a
      shell completion script \\
      \texttt{--man}                &       &         & Generate a man page \\
      \texttt{--args}               &       &         & View
      arguments passed to tex-fmt \\
      \texttt{--help}               & \texttt{-h}  &         & Print help \\
      \texttt{--version}            & \texttt{-V}  &         & Print version \\
      \hline
    \end{tabular}
  \end{center}
\end{table}

\subsection{Configuration file options}

The following arguments can be provided in \texttt{tex-fmt.toml}.
The first example in each row is the default value.

\begin{table}[H]
  \begin{center}
    \begin{tabular}{|l|l|l|l|}
      \hline
      \multicolumn{1}{|c|}{Option}
      & \multicolumn{1}{|c|}{Type}
      & \multicolumn{1}{|c|}{Examples}
      & \multicolumn{1}{|c|}{Description} \\
      \hline
      \texttt{check}          & bool     & \texttt{false}
      & Check formatting, do not modify files \\
      \texttt{print}          & bool     & \texttt{false}
      & Print to stdout, do not modify files \\
      \texttt{fail-on-change} & bool     & \texttt{false}
      & Fail if files are modified \\
      \texttt{wrap}           & bool     & \texttt{true}
      & Wrap long lines \\
      \texttt{wraplen}        & int      & \texttt{80}, \texttt{100}
      & Line length for wrapping \\
      \texttt{wrapmin}        & int      & \texttt{70}, \texttt{90}
      & Target minimum length for line wrapping \\
      \texttt{tabsize}        & int      & \texttt{2}, \texttt{4}
      & Number of characters to use as tab size \\
      \texttt{tabchar}        & str      & \texttt{"space"},
      \texttt{"tab"}     & Character to use for indentation \\
      \texttt{stdin}          & bool     & \texttt{false}
      & Process stdin as a single file, output to stdout \\
      \texttt{lists}          & arr[str] & \texttt{[]},
      \texttt{["myitemize"]}  & Extra list environments to be
      formatted as \texttt{itemize} \\
      \texttt{verbatims}      & arr[str] & \texttt{[]},
      \texttt{["myverbatim"]} & Extra verbatim environments \\
      \texttt{no-indent-envs} & arr[str] & \texttt{[]},
      \texttt{["mydocument"]} & Environments which are not indented \\
      \texttt{wrap-chars}     & arr[str] & \texttt{[]},
      \texttt{["。"]}         & Characters after which lines may be wrapped \\
      \texttt{verbosity}      & str      & \texttt{"warn"},
      \texttt{"error"}    & Verbosity level for terminal logging \\
      \hline
    \end{tabular}
  \end{center}
\end{table}

\end{document}
