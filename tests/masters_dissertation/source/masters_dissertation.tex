\documentclass[12pt,draft]{ociamthesis}

%TC:ignore

% PDF Version
%\pdfminorversion=7

% general typesetting
\usepackage[utf8]{inputenc}
\usepackage[british]{babel}
\usepackage{microtype}
\usepackage[table]{xcolor}

% lengths
\usepackage{etoolbox}
\usepackage{setspace}

% mathematics typesetting
\usepackage{amsmath}
\usepackage{amssymb}
\usepackage{amsthm}
\usepackage{mathtools}
\usepackage{dsfont}

% headers
\usepackage{fancyhdr}
\fancyhead{}
\renewcommand{\headrulewidth}{0pt}

% Lof, LoT
\usepackage{tocloft}
\setlength{\cftfigindent}{0pt}
\setlength{\cfttabindent}{0pt}

% algorithms
\usepackage[boxruled,
linesnumbered,
commentsnumbered,
algochapter,
]{algorithm2e}

% graphics
\usepackage{graphicx}
\usepackage{float}
\usepackage{subcaption}

% draft options
%\usepackage{ifdraft}
%\ifoptiondraft{
%\usepackage{draftwatermark}
%\SetWatermarkText{DRAFT}
%\SetWatermarkScale{6}
%\SetWatermarkColor[rgb]{1,0.9,0.9}
%\usepackage{showframe}
%\usepackage{layout}
%}{}

% hyperlinks
\usepackage[plainpages=false,draft=false
,hidelinks
]{hyperref}
\usepackage{cite}

% glossary
\usepackage[nopostdot,nonumberlist]{glossaries}

%TC:endignore
% suppress pdf warnings
%\pdfsuppresswarningpagegroup=1

\title{Motif\hspace*{0.05cm}-Based Spectral Clustering\\[1ex]
of Weighted Directed Networks}

\author{William George Underwood}
\college{Department of Statistics}

\renewcommand{\submittedtext}{}
\degree{Part C Dissertation in Mathematics \& Statistics}
\degreedate{Trinity 2019}
%TC:ignore

\theoremstyle{plain}
\newtheorem{theorem}{Theorem}[chapter]
\newtheorem{proposition}{Proposition}[chapter]

\theoremstyle{definition}
\newtheorem{definition}{Definition}[chapter]
\newtheorem{example}{Example}[chapter]
\newtheorem{prf}{Proof}[chapter]

\theoremstyle{remark}
\newtheorem*{remark}{Remark}
\newtheorem*{notation}{Notation}

% algorithms
\DontPrintSemicolon

% input output definitions
\makeatletter
\renewcommand{\SetKwInOut}[2]{%
\sbox\algocf@inoutbox{\KwSty{#2}\algocf@typo:}%
\expandafter\ifx\csname InOutSizeDefined\endcsname\relax%
\newcommand\InOutSizeDefined{}\setlength{\inoutsize}{\wd\algocf@inoutbox}%
\sbox\algocf@inoutbox{\parbox[t]{\inoutsize}%
{\KwSty{#2}\algocf@typo:\hfill}~}%
\setlength{\inoutindent}{\wd\algocf@inoutbox}%
\else% else keep the larger dimension
\ifdim\wd\algocf@inoutbox>\inoutsize%
\setlength{\inoutsize}{\wd\algocf@inoutbox}%
\sbox\algocf@inoutbox{\parbox[t]{\inoutsize}%
{\KwSty{#2}\algocf@typo:\hfill}~}%
\setlength{\inoutindent}{\wd\algocf@inoutbox}%
\fi%
\fi% the dimension of the box is now defined.
\algocf@newcommand{#1}[1]{%
\ifthenelse{\boolean{algocf@inoutnumbered}}{\relax}{\everypar={\relax}}%
{\let\\\algocf@newinout\hangindent=\inoutindent\hangafter=1\parbox[t]%
{\inoutsize}{\KwSty{#2}\algocf@typo:\hfill}~##1\par}%
\algocf@linesnumbered% reset the numbering of the lines
}}%
\makeatother

% keywords
\SetKwInOut{Input}{Input}
\SetKwInOut{Output}{Output}
\SetKw{In}{in}
\SetKwProg{Function}{function}{:}{}

% algorithm comment styles
\newcommand\commfont[1]{\rmfamily{#1}}
\SetCommentSty{commfont}
\SetKwComment{Comm}{$\rhd\ $}{}

% line spacing
\AtBeginEnvironment{algorithm}{\setstretch{1.15}}

% glossaries
\setlength{\glsdescwidth}{0.92\hsize}
\newglossarystyle{mystyle}{%
\setglossarystyle{long}%
\renewenvironment{theglossary}%
\begin{longtable}{@{}p{2cm}p{\glsdescwidth}}%
\end{longtable}%
}
\makeglossaries

% macros
\newcommand\bb[1]{\mathbb{#1}}
\newcommand\ca[1]{\mathcal{#1}}
\newcommand\Aut{\mathrm{Aut}}

% for inputting tables
\makeatletter\let\expandableinput\@@input\makeatother

%TC:endignore
% Glossary

\newglossaryentry{MAM}{name=MAM, description={Motif adjacency matrix}}
\newglossaryentry{DSBM}{name=DSBM,
description={Directed stochastic block model}}
\newglossaryentry{BSBM}{name=BSBM,
description={Bipartite stochastic block model}}
\newglossaryentry{Ncut}{name=Ncut, description={Normalised cut}}
\newglossaryentry{ARI}{name=ARI, description={Adjusted Rand Index}}

\glsaddall

\begin{document}

%TC:ignore
% give sufficient line spacing for comment markup
\baselineskip=18pt plus1pt

% set how many section levels get numbers and appear in the contents
\setcounter{secnumdepth}{3}
\setcounter{tocdepth}{2}

% do not hyphenate short words
\lefthyphenmin4
\righthyphenmin4

\pagenumbering{Alph}
%TC:endignore
\maketitle
\clearpage{}
\begin{abstract}

Clustering is an essential technique for network analysis, with applications
in a diverse range of fields. Although spectral clustering is a popular and
effective method, it fails to consider higher-order structure and can perform
poorly on directed networks. We aim to address these shortcomings by
exploring motif-based spectral clustering methods. We present new matrix
formulae for motif adjacency matrices, and a motif-based approach for
clustering bipartite networks. Comprehensive experimental results from both
synthetic and real data demonstrate the effectiveness of our techniques on a
variety of networks. We conclude that motif-based spectral clustering is a
valuable tool for analysis of directed and bipartite weighted networks, which
is also scalable and easy to implement.

\end{abstract}
\clearpage{}
%TC:ignore

\pagenumbering{arabic}
\begin{romanpages}

\tableofcontents

\newpage
\listoffigures

\newpage
\listoftables

\begingroup
\let\cleardoublepage\relax
\let\clearpage\relax
%\printglossary[title=Abbreviations, style=mystyle]
\endgroup

\end{romanpages}

% fancy headers
\pagestyle{fancy}
\renewcommand{\chaptermark}[1]{\markboth{#1}{}}
\fancyhead[RO]{\itshape{\nouppercase{Chapter \thechapter : \leftmark}}}

%TC:endignore
\clearpage{}
\chapter{Introduction}

% Importance of network analysis in the modern world
Networks are ubiquitous in modern society; from the internet and online blogs
to protein interactions and human migration, we are surrounded by inherently
connected structures~\cite{kolaczyk2014statistical}.
The mathematical and statistical analysis of networks is therefore a very
important area of modern research, with applications in a diverse range of
fields including biology~\cite{albert2005scale},
chemistry~\cite{jacob2018statistics}, physics~\cite{newman2008physics} and
sociology~\cite{adamic2005political}.

% Clustering is a core technique
A common problem in network analysis is that of
\emph{clustering}~\cite{schaeffer2007graph}.
Network clustering refers to the division of a network into several parts so
that objects in the same part are similar, while those in different parts are
dissimilar.

%Spectral methods are good
Spectral methods for network clustering have a long and successful
history~\cite{cheeger1969lower,donath1972algorithms,guattery1995performance},
and have become increasingly popular in recent years.
These techniques exhibit many attractive properties including generality, ease
of implementation and scalability~\cite{von2007tutorial}.

% Shortcomings of spectral methods
However traditional spectral methods have shortcomings, particularly involving
their inability to consider higher-order network
structures~\cite{benson2016higher}, and their insensitivity to edge
direction~\cite{DirectedClustImbCuts}. These weaknesses can lead to
unsatisfactory results, especially when working with directed networks.
Motif-based spectral methods have proven more effective for clustering directed
networks on the basis of higher-order
structures~\cite{tsourakakis2017scalable}, with the introduction of the
\emph{motif adjacency matrix} (MAM).

% Problems we want to solve
In this dissertation we will explore motif-based spectral clustering methods
with a focus on addressing these shortcomings for weighted directed networks.
Our main contributions include a collection of new matrix-based formulae for
MAMs on weighted directed networks, and a motif-based approach for clustering
bipartite networks. We also provide comprehensive experimental results both
from synthetic data (stochastic block models) and from real-world network data.

\section*{Dissertation layout}

In Chapter~\ref{chap:graphs} we describe our graph-theoretic framework which
provides a natural model for real-world weighted directed networks.
We define motifs and instances, and then state and prove new matrix-based
formulae for MAMs.
%
In Chapter~\ref{chap:spectral} we provide a summary of random-walk spectral
clustering and discuss techniques for cluster extraction and evaluation.
We state the algorithms for both traditional and motif-based spectral
clustering.
%
In Chapter~\ref{chap:motif} we introduce directed stochastic block models
(DSBMs), a family of generative models for directed networks, and evaluate the
performance of motif-based clustering both on synthetic data and on real data
(US Political Blogs network, US Migration network).
%
In Chapter~\ref{chap:bipartite} we propose a motif-based approach for
clustering bipartite graphs and introduce bipartite stochastic block models
(BSBMs), a family of generative models for bipartite networks. We again provide
experimental results both on synthetic data and on real data (American
Revolution network, Unicode Languages network).
%
Finally in Chapter~\ref{chap:conclusions} we present our conclusions, along
with a discussion about limitations and potential extensions of our work.

\clearpage{}
\clearpage{}
\chapter{Graphs and Motifs} \label{chap:graphs}

We describe our graph-theoretic framework for network analysis and give
matrix-based formulae for motif adjacency matrices (MAMs).
In Section~\ref{sec:graphs_graph_definitions} we outline basic concepts
relating to graphs and motifs.
In Section~\ref{sec:graphs_adj_and_ind_matrices} we define the adjacency and
indicator matrices of a graph.
In Section~\ref{sec:graphs_motif_adj_matrices} we introduce MAMs and present
the main results of this chapter,
Proposition~\ref{prop:motif_adj_matrix_formula} and
Proposition~\ref{prop:motif_adj_matrix_computation}.

\section{Graph definitions} \label{sec:graphs_graph_definitions}

Graph notation is notoriously inconsistent in the literature
\cite{intro_to_graph_theory}, so we begin by giving all of the relevant
notation and definitions.

\begin{definition}[Graphs]
A \emph{graph} is a triple $\ca{G} = (\ca{V,E},W)$ where $\ca{V}$ is the
\emph{vertex set}, $\ca{E} \subseteq \left\{ (i,j) : i,j \in \ca{V}, i \neq j
\right\}$ is the \emph{edge set} and $W\colon \ca{E} \to (0,\infty)$ is the
\emph{weight map}.
\end{definition}

\begin{remark}
We consider weighted directed graphs without self-loops or multiple edges. We
can extend to undirected graphs by replacing undirected edges with
bidirectional edges. Where it is not relevant, we may sometimes omit the
weight map $W$.
\end{remark}

\begin{definition}[Underlying edges]
Let $\ca{G} = (\ca{V,E})$ be a graph. Its \emph{underlying edges} are
$\bar{\ca{E}} \vcentcolon = \big\{ \{i,j\} : (i,j) \in \ca{E} \big\}$.
\end{definition}

\begin{definition}[Subgraphs]
A graph $\ca{G'} = (\ca{V',E'})$ is a \emph{subgraph} of a graph $\ca{G} =
(\ca{V,E})$ (write $\ca{G'} \leq \ca{G}$) if $\ca{V'} \subseteq \ca{V}$ and
$\ca{E'} \subseteq \ca{E}$. It is an \emph{induced subgraph} (write $\ca{G'}
< \ca{G}$) if further $\ca{E'} = \ca{E} \cap ( \ca{V'} \times \ca{V'} )$.
\end{definition}

\begin{definition}[Connected components]
Let $\ca{G} = (\ca{V,E})$ be a graph. The \emph{connected components} of
$\ca{G}$ are the partition $\ca{C}$ generated by the transitive closure of
the relation $\sim$ on $\ca{V}$ defined by $i \sim j \iff \{i,j\} \in
\bar{\ca{E}}$. We say $\ca{G}$ is (weakly) \emph{connected} if $|\ca{C}| =
1$.
\end{definition}

\begin{definition}[Graph isomorphisms]
A graph $\ca{G'} = (\ca{V',E'})$ is \emph{isomorphic} to a graph $\ca{G} =
(\ca{V,E})$ (write $\ca{G'} \cong \ca{G}$) if there is a bijection
$\phi\colon \ca{V'} \rightarrow \ca{V}$ with $(u,v) \in \ca{E'} \iff
\big(\phi(u), \phi(v) \big) \in \ca{E}$.
An isomorphism from a graph to itself is called an \emph{automorphism}.
\end{definition}

\begin{definition}[Motifs and anchor sets]
A \emph{motif} is a pair $(\ca{M,A})$ where $\ca{M} = (\ca{V_M,E_M})$ is a
connected graph with $\ca{V_M} = \{ 1, \ldots, m \}$ for some small $m \geq
2$, and $\ca{A} \subseteq \ca{V_M}$ with $|\ca{A}| \geq 2$ is an \emph{anchor
set}. If $\ca{A} \neq \ca{V_M}$ we say the motif is \emph{anchored}, and if
$\ca{A=V_M}$ we say it is \emph{simple}.
\end{definition}

\begin{remark}
Anchor sets~\cite{benson2016higher} specify which r\^oles vertices play in
the motif, and are crucial for defining the collider and expander motifs
given in Section~\ref{sec:coll_expa}. When an anchor set is not given, it is
assumed that the motif is simple. Figure~\ref{fig:motif_definitions_directed}
shows all simple motifs (up to isomorphism) on at most three vertices.
\end{remark}

\begin{definition}[Instances]
Let $\ca{G}$ be a graph and $(\ca{M,A})$ a motif. We say that $\ca{H}$ is a
\emph{functional instance} of $\ca{M}$ in $\ca{G}$ if $\ca{M} \cong \ca{H}
\leq \ca{G}$. We say that $\ca{H}$ is a \emph{structural instance} of
$\ca{M}$ in $\ca{G}$ if $\ca{M} \cong \ca{H} < \ca{G}$.
\end{definition}

\begin{definition}[Anchored pairs]
Let $\ca{G}$ be a graph and $(\ca{M,A})$ a motif. Suppose $\ca{H}$ is an
instance of $\ca{M}$ in $\ca{G}$. Define the \emph{anchored pairs of the
instance} $\ca{H}$ as
$$ \ca{A(H)} \vcentcolon = \big\{ \{\phi(i),\phi(j)\} : i,j \in \ca{A}, \ i
\neq j, \ \phi \textrm{ is an isomorphism from } \ca{M} \textrm{ to } \ca{H}
\big\}\,.$$

\end{definition}

\begin{remark}
Example~\ref{ex:instances} demonstrates functional and structural instances.
Note that $\{i,j\} \in \ca{A(H)}$ if and only if $\ca{H}$ appears in $\ca{G}$
as an instance of $\ca{M}$ with $i \neq j$ co-appearing in the image of
$\ca{A}$ under isomorphism. The motivation for this is that clustering
methods should avoid separating vertices which appear as an anchored pair.
\end{remark}
%
\begin{figure}[H]
\centering
%\includegraphics[scale=0.7,draft=false]{%
%../tikz/motif_definitions_directed/motif_definitions_directed.pdf}
\caption{All simple motifs on at most three vertices}
\label{fig:motif_definitions_directed}
\end{figure}

\section{Adjacency and indicator matrices}
\label{sec:graphs_adj_and_ind_matrices}

Adjacency matrices provide a useful data structure for representing graphs and
have many uses in calculating graph properties \cite{bapat2010graphs}.  We
define several variants of the adjacency matrix, which appear in
Proposition~\ref{prop:motif_adj_matrix_formula} and
Table~\ref{tab:motif_adj_mat_table}.

\begin{definition}[Adjacency matrices]
Let $\ca{G} = (\ca{V,E},W)$ be a graph with vertex set $\ca{V} = \{1, \ldots,
n \}$. The \emph{adjacency matrix, single-edge adjacency matrix} and
\emph{double-edge adjacency matrix} of $\ca{G}$ are respectively the $n
\times n$ matrices
\begin{align*}
G_{i j} &\vcentcolon= W((i,j)) \ \bb{I} \{ (i,j) \in \ca{E} \}\,, \\
(G_\mathrm{s})_{i j} &\vcentcolon= W((i,j)) \ \bb{I} \{ (i,j) \in \ca{E}
\textrm{ and } (j,i) \notin \ca{E} \}\,, \\
(G_\mathrm{d})_{i j} &\vcentcolon= \big( W((i,j)) + W((j,i)) \big) \ \bb{I}
\{ (i,j) \in \ca{E} \textrm{ and } (j,i) \in \ca{E} \}\,.
\end{align*}
\end{definition}

\begin{definition}[Indicator matrices]
Let $\ca{G} = (\ca{V,E},W)$ be a graph with vertex set $\ca{V} = \{1, \ldots,
n \}$. The \emph{indicator matrix, single-edge indicator matrix, double-edge
indicator matrix, missing-edge indicator matrix} and \emph{vertex-distinct
indicator matrix} of $\ca{G}$ are respectively the $n \times n$ matrices
\begin{align*}
J_{i j} &\vcentcolon= \bb{I} \{ (i,j) \in \ca{E} \}\,, \\
(J_\mathrm{s})_{i j} &\vcentcolon= \bb{I} \{ (i,j) \in \ca{E} \textrm{ and }
(j,i) \notin \ca{E} \}\,, \\
(J_\mathrm{d})_{i j} &\vcentcolon= \bb{I} \{ (i,j) \in \ca{E} \textrm{ and }
(j,i) \in \ca{E} \}\,, \\
(J_0)_{i j} &\vcentcolon= \bb{I} \{ (i,j) \notin \ca{E} \textrm{ and } (j,i)
\notin \ca{E} \textrm{ and } i \neq j \}\,, \\
(J_\mathrm{n})_{i j} &\vcentcolon= \bb{I} \{ i \neq j \}\,.
\end{align*}
\end{definition}

\section{Motif adjacency matrices} \label{sec:graphs_motif_adj_matrices}

The central object in motif-based spectral clustering is the \emph{motif
adjacency matrix} (MAM) \cite{benson2016higher}, which serves as a similarity
matrix for spectral clustering (Chapter~\ref{chap:spectral}).
We provide here our main results:
Proposition~\ref{prop:motif_adj_matrix_formula} gives a computationally useful
formula for MAMs, and Proposition~\ref{prop:motif_adj_matrix_computation} gives
a complexity analysis of this formula.

\pagebreak

\subsection{Definitions}

\begin{definition}[Motif adjacency matrices] \label{def:motif_adj_matrices}
%
Let $\ca{G} = (\ca{V,E},W)$ be a graph with $n$ vertices and let $\ca{(M,A)}$
be a motif. The \emph{functional} and \emph{structural motif adjacency
matrices} (MAMs) of $\ca{(M,A)}$ in $\ca{G}$ are respectively the $n \times
n$ matrices
%
\begin{align*}
M^\mathrm{func}_{i j} &\vcentcolon= \frac{1}{|\ca{E_M}|} \sum_{\ca{M} \cong
\ca{H} \leq \ca{G}} \bb{I} \big\{ \{i,j\} \in \ca{A}(\ca{H}) \big\} \sum_{e
\in \ca{E_H}} W(e)\,, \\
M^\mathrm{struc}_{i j} &\vcentcolon= \frac{1}{|\ca{E_M}|} \sum_{\ca{M} \cong
\ca{H} < \ca{G}} \bb{I} \big\{ \{i,j\} \in \ca{A}(\ca{H}) \big\} \sum_{e
\in \ca{E_H}} W(e)\,.
\end{align*}
\end{definition}

\begin{remark}
Example~\ref{ex:motif_adj_matrices} gives a simple illustration of
calculating an MAM.
When $W \equiv 1$ and $\ca{M}$ is simple, the (functional or structural) MAM
entry $M_{i j} \ (i \neq j)$ simply counts the (functional or structural)
instances of $\ca{M}$ in $\ca{G}$ containing $i$ and $j$.
When $\ca{M}$ is not simple, $M_{i j}$ counts only those instances with anchor
sets containing both $i$ and $j$.
MAMs are always symmetric, since the only dependency on $(i,j)$ is via the
unordered set $\{i,j\}$.
\end{remark}

\subsection{Computation} \label{sec:graphs_computation}

In order to state Propositions \ref{prop:motif_adj_matrix_formula}
and~\ref{prop:motif_adj_matrix_computation}, we need one more definition.

\begin{definition}[Anchored automorphism classes]
Let $(\ca{M,A})$ be a motif.
Let $S_\ca{M}$ be the set of permutations on $ \ca{V_M} = \{ 1, \ldots, m \}$
and define the \emph{anchor-preserving permutations} $S_\ca{M,A} = \{ \sigma
\in S_\ca{M} : \{1,m\} \subseteq \sigma(\ca{A}) \}$.
Let $\sim$ be the equivalence relation defined on $S_\ca{M,A}$ by: $\sigma
\sim \tau \iff \tau^{-1} \sigma$ is an automorphism of $\ca{M}$.
Finally the \emph{anchored automorphism classes} are the quotient set
$S_\ca{M,A}^\sim \vcentcolon= S_\ca{M,A} \ \big/ \sim$\,.
\end{definition}

\begin{proposition}[MAM formula] \label{prop:motif_adj_matrix_formula}
Let $\ca{G} = (\ca{V,E},W)$ be a graph with vertex set
${\ca{V}=\{1,\ldots,n\}}$ and let $(\ca{M,A})$ be a motif on $m$ vertices.
Then for any $i,j \in \ca{V}$ and with $k_1 = i$, $k_m = j$, the functional
and structural MAMs of $\ca{(M,A)}$ in $\ca{G}$ are given by
%
%
\begin{align*}
M^\mathrm{func}_{i j} &= \frac{1}{|\ca{E_M}|} \sum_{\sigma \in
S_\ca{M,A}^\sim} \ \sum_{\{k_2, \ldots, k_{m-1}\} \subseteq \ca{V}} \
J^\mathrm{func}_{\mathbf{k},\sigma} \
G^\mathrm{func}_{\mathbf{k},\sigma}\,, &(1) \\
M^\mathrm{struc}_{i j} &= \frac{1}{|\ca{E_M}|} \sum_{\sigma \in
S_\ca{M,A}^\sim} \ \sum_{\{k_2, \ldots, k_{m-1}\} \subseteq \ca{V}} \
J^\mathrm{struc}_{\mathbf{k},\sigma} \
G^\mathrm{struc}_{\mathbf{k},\sigma}\,, &(2)
\end{align*}
%
where
%
\begin{align*}
\ca{E}_\ca{M}^0 &\vcentcolon= \{ (u,v) : 1 \leq u < v \leq m : (u,v) \notin
\ca{E_M}, (v,u) \notin \ca{E_M} \}\,, \\
\ca{E}_\ca{M}^\mathrm{s} &\vcentcolon= \{ (u,v) : 1 \leq u < v \leq m :
(u,v) \in \ca{E_M}, (v,u) \notin \ca{E_M} \}\,, \\
\ca{E}_\ca{M}^\mathrm{d} &\vcentcolon= \{ (u,v) : 1 \leq u < v \leq m :
(u,v) \in \ca{E_M}, (v,u) \in \ca{E_M} \}\,,
\end{align*}
%
are respectively the missing edges, single edges and double edges of
$\ca{E_M}$, and
%
%TC:ignore
\begin{alignat*}{3}
%
J^\mathrm{func}_{\mathbf{k},\sigma}
& \vcentcolon= \prod_{\ca{E}_\ca{M}^0} (J_\mathrm{n})_{k_{\sigma
u},k_{\sigma v}}
&& && \prod_{\ca{E}_\ca{M}^\mathrm{s}} J_{k_{\sigma u},k_{\sigma v}}
\prod_{\ca{E}_\ca{M}^\mathrm{d}} (J_\mathrm{d})_{k_{\sigma u},k_{\sigma
v}}\,, \\
%
G^\mathrm{func}_{\mathbf{k},\sigma}
& \vcentcolon= \sum_{\ca{E}_\ca{M}^\mathrm{s}} G_{k_{\sigma u},k_{\sigma
v}}
&& + && \sum_{\ca{E}_\ca{M}^\mathrm{d}} (G_\mathrm{d})_{k_{\sigma
u},k_{\sigma v}}\,, \\
%
J^\mathrm{struc}_{\mathbf{k},\sigma}
& \vcentcolon= \prod_{\ca{E}_\ca{M}^0} (J_0)_{k_{\sigma u},k_{\sigma v}}
&& && \prod_{\ca{E}_\ca{M}^\mathrm{s}} (J_\mathrm{s})_{k_{\sigma
u},k_{\sigma v}}
\prod_{\ca{E}_\ca{M}^\mathrm{d}} (J_\mathrm{d})_{k_{\sigma u},k_{\sigma
v}}\,, \\
%
G^\mathrm{struc}_{\mathbf{k},\sigma}
&\vcentcolon= \sum_{\ca{E}_\ca{M}^\mathrm{s}} (G_\mathrm{s})_{k_{\sigma
u},k_{\sigma v}}
&& + && \sum_{\ca{E}_\ca{M}^\mathrm{d}} (G_\mathrm{d})_{k_{\sigma
u},k_{\sigma v}}\,.
%
\end{alignat*}
%TC:endignore
\end{proposition}
%
\begin{proof}
See Proof~\ref{proof:motif_adj_matrix_formula}.
\end{proof}

\begin{proposition}[Complexity of MAM formula]
\label{prop:motif_adj_matrix_computation}
Suppose that ${m \leq 3}$, and the adjacency matrix $G$ of $\ca{G}$ is known.
Then computing adjacency and indicator matrices and calculating an MAM using
Equations $(1)$ and $(2)$ in Proposition~\ref{prop:motif_adj_matrix_formula}
involves at most 18 matrix multiplications, 22 entry-wise multiplications and
21 additions of (typically sparse) $n \times n$ matrices.
\end{proposition}

\begin{proof}
See Proof~\ref{proof:motif_adj_matrix_computation}.
\end{proof}

Hence for motifs on at most three vertices and with sparse adjacency matrices,
Proposition~\ref{prop:motif_adj_matrix_formula} gives a fast and parallelisable
matrix-based procedure for computing MAMs. In practice, additional symmetries
of the motif often allow computation with even fewer matrix operations,
demonstrated in Example~\ref{ex:motif_adj_calc}.

A list of such MAM formulae for all simple motifs on at most three vertices (up
to isomorphism), as well as for the \emph{collider} and \emph{expander} motifs
(Section~\ref{sec:coll_expa}), is given in Table~\ref{tab:motif_adj_mat_table}.
These formulae are generalisations of those stated in Table S6 in the
supplementary materials for \cite{benson2016higher}, in an incomplete list of
only \emph{structural} MAMs of \emph{unweighted} graphs. Note that the
functional MAM formula for the two-vertex motif $\ca{M}_\mathrm{s}$ yields the
symmetrised adjacency matrix $M = G + G^\top$ which is used for traditional
spectral clustering (Section~\ref{sec:spectral_overview}). The question of
whether to use functional or structural MAMs for motif-based spectral
clustering will be addressed in Section~\ref{sec:spectral_motifrwspectclust}.

\clearpage{}
\clearpage{}
\chapter{Spectral Clustering} \label{chap:spectral}

We provide a summary of traditional random-walk spectral clustering and show
how it applies to motif-based clustering.
This chapter mostly follows the relevant sections in the tutorial by
U.~Von~Luxburg~\cite{von2007tutorial}, which provides further explanations and
proofs.
In Section~\ref{sec:spectral_overview} we give an overview of the spectral
clustering procedure.
In Section~\ref{sec:spectral_laplacians} we define the random-walk Laplacian
and state some of its useful properties (Proposition~\ref{prop:laplacian}).
In Section~\ref{sec:spectral_graph_cut} we introduce normalised cut (Ncut) as
an objective function for graph partitioning.
In Section~\ref{sec:spectral_cluster_extraction} we explore methods of
extracting clusters from $\bb{R}^l$-valued embeddings, and in
Section~\ref{sec:spectral_algs} we present the algorithms for both traditional
and motif-based random-walk spectral clustering.

\section{Overview of spectral clustering} \label{sec:spectral_overview}

Suppose $x_1, \ldots, x_n$ are data points with some associated symmetric
similarity matrix $M$ with ${M_{i j} = \mathrm{similarity}(x_i,x_j)}$.
The intuitive aim of clustering is to find a partition $\ca{P}_1, \ldots,
\ca{P}_k$ of $\{ x_1, \ldots, x_n \}$ which places similar points in the same
group and dissimilar points in different groups.
Where other methods such as $k$-means++  \cite{arthur2007k} and GMM clustering
\cite{duda1973pattern} demand some further structure on $x_i$ (such as taking
values in $\bb{R}^l$), spectral clustering has no such requirements.

In the context of \emph{undirected} graph clustering, the data points are the
vertices of the graph, and a similarity matrix is provided by the graph's
adjacency matrix $G$.
To cluster directed graphs, the adjacency matrix must first be symmetrised,
traditionally by the transformation $M = G + G^\top$
\cite{Meila2007ClusteringBW}.
This symmetrisation ignores information about edge direction and higher-order
structures; and can lead to poor performance, as will be seen in
Section~\ref{sec:motif_asymm_dsbms}.

Spectral clustering consists of two steps. Firstly, eigendecomposition of a
Laplacian matrix embeds the vertices into $\bb{R}^{l}$. The $k$ clusters are
then extracted from this space.

\section{Graph Laplacians} \label{sec:spectral_laplacians}

The Laplacians of an undirected graph are a family of matrices which play a
central r\^ole in spectral clustering. While many different graph Laplacians
are available, we focus in this dissertation on just the \emph{random-walk
Laplacian}, for reasons concerning objective functions, consistency and
computation \cite{von2007tutorial, luxburg2004convergence}.

\begin{definition}
Let $\ca{G}$ be an undirected graph with (symmetric) adjacency matrix $G$. The
\emph{random-walk Laplacian matrix} of $\ca{G}$ is
$$ L_\mathrm{rw} \vcentcolon= I - D^{-1} G $$
where $I$ is the identity and $D_{ii} \vcentcolon= \sum_j G_{i j}$ is the
diagonal matrix of weighted degrees.
\end{definition}

\begin{remark}
$D^{-1} G$ is the transition matrix of a random walk on the vertex set
$\ca{V}$
where the probability of the transition $v_i \to v_j$ is proportional to
$G_{i j}$.
\end{remark}

\begin{proposition}[Properties of the random-walk Laplacian]
\label{prop:laplacian}
$L_\mathrm{rw}$ is positive semi-definite with eigenvalues $0 = \lambda_1 \leq
\cdots \leq \lambda_n$.
The multiplicity $k$ of the eigenvalue $0$ is equal to the number of connected
components $\ca{P}_1, \ldots, \ca{P}_k$ of $\ca{G}$.
The eigenspace of the eigenvalue $0$ is spanned by the indicator vectors on
these components; $ \bb{I}_{\ca{P}_1}, \ldots, \bb{I}_{\ca{P}_k} $.
\end{proposition}

\begin{proof}
See \cite{von2007tutorial}.
\end{proof}

\section{Graph cuts} \label{sec:spectral_graph_cut}

Graph cuts provide objective functions which we seek to minimise while
clustering the vertices of a graph.
We look at the normalised cut and its relationship with the random-walk
Laplacian.

\begin{definition}
Let $\ca{G}$ be a graph. Let $ \ca{P}_1, \ldots, \ca{P}_k $ be a partition of
$\ca{V}$. Then the \emph{normalised cut} \cite{shi2000normalized} of $\ca{G}$
with respect to $ \ca{P}_1, \ldots, \ca{P}_k $ is
%
$$ \mathrm{Ncut}_\ca{G}(\ca{P}_1, \ldots, \ca{P}_k) \vcentcolon= \frac{1}{2}
\sum_{i=1}^k
\frac{ \mathrm{cut}(\ca{P}_i,\bar{\ca{P}_i}) }{ \mathrm{vol}(\ca{P}_i) } $$
%
where $ \mathrm{cut}(\ca{P}_i,\bar{\ca{P}_i})
\vcentcolon= \sum_{u \in \ca{P}_i, \, v \in \ca{V} \setminus \ca{P}_i}
G_{u v}$
and $\mathrm{vol}(\ca{P}_i) \vcentcolon= \sum_{u \in \ca{P}_i} D_{u u}$.

\end{definition}

\begin{remark}
More desirable partitions have a lower Ncut value; the numerators penalise
partitions which cut a large number of heavily weighted edges, and the
denominators penalise partitions which have highly imbalanced cluster sizes.
\end{remark}

It can be shown \cite{von2007tutorial} that minimising Ncut over partitions
$ \ca{P}_1, \ldots, \ca{P}_k $ is equivalent to finding the cluster indicator
matrix $H \in \bb{R}^{n \times k}$ minimising
$$ \mathrm{Tr} \big( H^\top (D-G) H \big) $$
subject to
$$ H_{i j} = \mathrm{vol}(\ca{P}_j)^{-\frac{1}{2}} \ \bb{I} \{ v_i \in \ca{P}_j
\}\,, \qquad (\dagger) $$
$$ H^\top D H = I\,. $$

Solving this problem is in general \textsf{NP}-hard \cite{wagner1993between}.
However, by dropping the constraint~$(\dagger)$ and applying the Rayleigh
Principle \cite{lutkepohl1996handbook}, we find that the solution to this
relaxed problem is that $H$ contains the first $k$ eigenvectors of
$L_\mathrm{rw}$ as columns \cite{von2007tutorial}.
In practice, to find $k$ clusters it is often sufficient to use only the first
$l < k$ eigenvectors of $L_\mathrm{rw}$.

\section{Cluster extraction} \label{sec:spectral_cluster_extraction}

Once Laplacian eigendecomposition has been used to embed the data into
$\bb{R}^l$, the clusters may be extracted using a variety of methods. We
propose $k$-means++ and eigenvector sweep as two appropriate techniques.

\subsection{\texorpdfstring{$k$}{k}-means++}

$k$-means++ \cite{arthur2007k} is a popular clustering algorithm for data in
$\bb{R}^l$.
It aims to minimise the within-cluster sum of squares, based on the standard
Euclidean metric on $\bb{R}^l$.
This makes it a reasonable candidate for clustering spectral data, since the
Euclidean metric corresponds to notions of `diffusion distance' in the original
graph \cite{nadler2006diffusion}.

\subsection{Eigenvector sweep} \label{sec:spectral_sweep}

Eigenvector sweep
(Algorithm~\ref{alg:eigenvector_sweep})~\cite{shi2000normalized} offers a more
principled technique for cluster extraction when $k=2$ clusters are required,
and a single eigenvector (usually the second eigenvector of $L_\mathrm{rw}$) is
available. It works by sorting the eigenvector and selecting a splitting point
to minimise the Ncut score of the partition generated.

\pagebreak

\begin{algorithm}[H]
\caption{Eigenvector sweep}
\label{alg:eigenvector_sweep}

\SetKwFunction{Main}{EigenvectorSweep}
\newcommand{\MainArgs}{$\ca{G}, x$}

\BlankLine
\Input{Graph $\ca{G}$, eigenvector $x$}
\Output{Partition $\ca{P}_1, \ca{P}_2$}
\BlankLine
\Function{\Main{\MainArgs}}{

$\hat{x} \leftarrow \mathtt{sort}(x)$ \;
$\mathrm{Score_{best}} \leftarrow \infty$ \;

\For{$i$ \In $1, \ldots, n-1$}{
$\ca{P} \leftarrow \{ \hat{x}_1, \ldots \hat{x}_i \}$ \;
$\mathrm{Score} \leftarrow \mathrm{Ncut}_\ca{G} (\ca{P}, \ca{V}
\setminus \ca{P})$ \;
\If{$\mathrm{Score} < \mathrm{Score_{best}}$}{
$\ca{P}_\mathrm{best} \leftarrow \ca{P}$ \;
$\mathrm{Score_{best}} \leftarrow \mathrm{Score}$ \;
}
}

$\ca{P}_1 \leftarrow \ca{P}_\mathrm{best}$ \;
$\ca{P}_2 \leftarrow \ca{V} \setminus \ca{P}_\mathrm{best}$ \;

\Return $\ca{P}_1, \ca{P}_2$
}
\end{algorithm}
\vspace*{0.5cm}

Figure~\ref{fig:eigenvector_sweep_network} shows a small network with vertices
labelled by position in the sorted second eigenvector $\hat{x}$ of
$L_\mathrm{rw}$. Figure~\ref{fig:eigenvector_sweep_profile} shows the `sweep
profile' of Ncut scores, which is minimised at the splitting point $i=5$. Hence
eigenvector sweep chooses the final partition $\ca{P}_1 = \{1, \ldots,5\}, \
\ca{P}_2 = \{6, \ldots,10\}$; as indicated by the vertex colours and dashed
line in Figure~\ref{fig:eigenvector_sweep_network}.
%
%
\begin{figure}[H]
\begin{subfigure}{.49\textwidth}
\centering
%\includegraphics[scale=0.4,draft=false]{%
%../tikz/eigenvector_sweep_network/eigenvector_sweep_network.pdf}
\caption{A small network}
\label{fig:eigenvector_sweep_network}
\end{subfigure}
%
\begin{subfigure}{.49\textwidth}
\centering
%\includegraphics[scale=0.4,draft=false]{%
%../../results/eigenvector_sweep/eigenvector_sweep_scores.pdf}
\caption{Sweep profile of the network}
\label{fig:eigenvector_sweep_profile}
\end{subfigure}
\caption{Eigenvector sweep selects a partition by minimising Ncut}
\label{fig:eigenvector_sweep}
\end{figure}
%

\subsection{Cluster evaluation}

When a graph has been clustered, we assign a score to the partition. If the
ground-truth clustering is available, we can compare it to our clustering using
the \emph{adjusted Rand index} (ARI) \cite{hubert1985comparing}. The ARI
between two clusterings has expected value $0$ under random cluster assignment,
and maximum value $1$ denoting perfect agreement between the clusterings. A
larger ARI indicates a more similar clustering.

If the ground-truth clustering is not available, we can use the objective
function Ncut. Clusterings with lower Ncut values partition the graph more
agreeably.

\section{Spectral clustering algorithms} \label{sec:spectral_algs}

We present the full random-walk spectral clustering algorithm and show how it
can be applied to motif-based random-walk spectral clustering.

\subsection{Random-walk spectral clustering}

Algorithm~\ref{alg:rwspectclust} gives random-walk spectral clustering
\cite{von2007tutorial}, which takes a symmetric connected adjacency matrix as
input. We use $k$-means++ rather than eigenvector sweep as the cluster
extraction method, due to its superior flexibility and computational speed. We
drop the first column of $H$ (the first eigenvector of $L_\mathrm{rw}$) since
although it should be constant and uninformative
(Proposition~\ref{prop:laplacian}), numerical imprecision may give unwanted
artefacts. It is worth noting that although the relaxation used in
Section~\ref{sec:spectral_graph_cut} is reasonable and often leads to good
approximate solutions of the Ncut problem, there are cases where it performs
poorly~\cite{guattery1998quality}. The Cheeger
inequality~\cite{chung2005laplacians} gives a bound on the error introduced by
this relaxation.

\vspace*{0.5cm}
\begin{algorithm}[H]
\caption{Random-walk spectral clustering}
\label{alg:rwspectclust}

\SetKwFunction{Main}{RWSpectClust}
\newcommand{\MainArgs}{$G,k,l$}

\BlankLine
\Input{Symmetric adjacency matrix $G$, number of clusters $k$, dimension
$l$}
\Output{Partition $\ca{P}_1, \ldots, \ca{P}_k$}
\BlankLine

\Function{\Main{\MainArgs}}{
Construct the weighted degree matrix $D_{ii} \leftarrow \sum_j G_{i j}$
\\
Construct the random walk Laplacian matrix $L_\mathrm{rw} \leftarrow
I-D^{-1}G$ \\
Let $H$ have the first $l$ eigenvectors of $L_\mathrm{rw}$ as columns
\\
Drop the first column of $H$ \\
Run $k$-means++ on the rows of $H$ with $k$ clusters to produce
$\ca{P}_1, \ldots, \ca{P}_k$ \\
\Return $\ca{P}_1, \ldots, \ca{P}_k$
}

\end{algorithm}

\subsection{Motif-based random-walk spectral clustering}
\label{sec:spectral_motifrwspectclust}

Algorithm~\ref{alg:motifrwspectclust} gives motif-based random-walk spectral
clustering.
Note that although $\ca{G}$ may be a connected graph, there is no guarantee
that the MAM is connected too.
Hence $M$ is restricted to its largest connected component $C$ before spectral
clustering is applied.
While this may initially seem to be a flaw with motif-based spectral clustering
(since not all vertices are assigned to a cluster), in fact it can be very
useful; restriction of $M$ can remove vertices which are in some sense not
`well connected' to the rest of the graph, which means that only a `core' set
of vertices are clustered.
This can result in Algorithm~\ref{alg:motifrwspectclust} making fewer
misclassifications with motif-based methods than with traditional spectral
clustering, as seen in Section~\ref{sec:motif_polblogs}.

There is ambiguity in whether to use functional or structural MAMs.
While the authors in~\cite{benson2016higher} opt for structural MAMs, we
propose to use functional MAMs, for a few reasons.
Firstly, note that $ 0 \leq M^\mathrm{struc}_{i j} \leq M^\mathrm{func}_{i j}$
for all $i,j \in \ca{V}$.
This implies that the largest connected component of $M^\mathrm{func}$ is
always at least as large as that of $M^\mathrm{struc}$, meaning that often more
vertices can be assigned to a cluster.
Secondly, we argue that functional instances are of more interest than
structural motifs, since they specify only `existence' rather than
`non-existence' of edges.
For consistency we will therefore use functional MAMs throughout our
experiments.

The most computationally expensive part of
Algorithm~\ref{alg:motifrwspectclust} is the calculation of the MAM using a
formula from Table~\ref{tab:motif_adj_mat_table}. We found this to be feasible
for graphs with up to around $n \approx 10 \, 000$ vertices. General notes on
hardware and software are given in Section~\ref{sec:notes_hardware}, and
timings for MAM computation across a range of graph sizes and sparsities are
available in Section~\ref{sec:notes_timing}.

\vspace*{0.5cm}
\begin{algorithm}[H]
\caption{Motif-based random-walk spectral clustering}
\label{alg:motifrwspectclust}

\SetKwFunction{Main}{MotifRWSpectClust}
\newcommand{\MainArgs}{$\ca{G},\mathcal{M},k,l$}

\BlankLine
\Input{Graph $\ca{G}$, motif $\ca{M}$, number of clusters $k$, dimension
$l$}
\Output{Partition $\ca{P}_1, \ldots, \ca{P}_k$}
\BlankLine

\Function{\Main{\MainArgs}}{
Construct the motif adjacency matrix $M$ of the graph $\ca{G}$ with
motif $\ca{M}$ \\
Let $\tilde{M}$ be $M$ restricted to its largest connected component,
$C$ \\
$\ca{P}_1, \ldots, \ca{P}_k \leftarrow$
\texttt{RWSpectClust($\tilde{M},k,l$)} \\
\Return $\ca{P}_1, \ldots, \ca{P}_k$
}

\end{algorithm}

\clearpage{}
\clearpage{}
\chapter{Motif-Based Clustering} \label{chap:motif}

We analyse the performance of motif-based random-walk spectral clustering on
both synthetic and real data.
In Section~\ref{sec:motif_dsbms} we propose a family of stochastic block models
and perform experiments with a variety of motifs and parameters.
In Section~\ref{sec:motif_polblogs} we analyse the US Political Blogs network
and in Section~\ref{sec:motif_migration} we present results from the US
Migration network.

\section{Directed stochastic block models} \label{sec:motif_dsbms}

We begin by describing \emph{directed stochastic block models} (DSBMs), a broad
class of generative models for directed graphs. A DSBM is characterised by a
block count $k$, a list of block sizes $(n_i)_{i=1}^k$ and a sparsity matrix $F
\in [0,1]^{k \times k}$. We define the cumulative block sizes $N_i =
\sum_{j=1}^i n_j$ with $N_0=0$, and the total graph size $N=N_k$. These are
used to construct the expected adjacency matrix $A \in [0,1]^{N \times N}$
given by $A_{i j} = F_{rs} \ \bb{I}\{i \neq j\}$ where $N_{r-1} < i \leq N_r$
and $N_{s-1} < j \leq N_s$. Finally a graph $\ca{G}$ is generated with
adjacency matrix entries $G_{i j} \sim \textrm{Ber}(A_{i j})$ sampled
independently. We say that a DSBM is \emph{symmetric} if $F$ is a symmetric
matrix.

This DSBM definition is similar to that given by \cite{DirectedClustImbCuts},
although we impose independence between all entries of the adjacency matrix,
allowing for bidirectional edges.

\subsection{Symmetric two-block DSBMs}

We define the \emph{symmetric two-block DSBM} as the DSBM with $k=2$,
$n_1=n_2=n$ and
$F =
\begin{psmallmatrix}
p & q \\ q & p
\end{psmallmatrix}$
where $p > q$. Figure~\ref{fig:sym_two_block_dsbm} illustrates the block
structure and sparsity matrix of this model. Thicker lines indicate existence
of edges with higher probability.

\begin{figure}[H]
\centering
%\includegraphics[scale=0.8,draft=false]{%
%../tikz/sym_two_block_dsbm/sym_two_block_dsbm.pdf}
\caption{Symmetric two-block DSBM block structure and sparsity matrix}
\label{fig:sym_two_block_dsbm}
\end{figure}

We test the performance of Algorithm~\ref{alg:motifrwspectclust} across various
motifs with parameters $k=l=2$ on this model.
Figure~\ref{fig:motifsym} shows violin plots over 20 trials of ARI against
motif, for different sets of parameters $n,p,q$.
Also shown is $|C|$, the average size of the largest connected component of
each MAM.
It can be seen that several motifs (such as $\ca{M}_5$ and $\ca{M}_9$) achieve
a similar ARI to the traditional spectral clustering technique given by the
symmetrised adjacency matrix $M=G+G^\top$ generated by the motif
$\ca{M}_\mathrm{s}$ (Table~\ref{tab:motif_adj_mat_table}).
However the strongly connected motifs (particularly $\ca{M}_4$) generate MAMs
with small connected components, especially when $\ca{G}$ is sparse, and hence
only cluster a subset of the vertices of $\ca{G}$.

\begin{figure}[H]
\begin{subfigure}{.49\textwidth}
\centering
%\includegraphics[scale=0.4,draft=false]{%
%../../results/motifsym/motifsym_1.pdf}
\caption{$n=50$, $p=0.3$, $q=0.2$}
\end{subfigure}
\begin{subfigure}{.49\textwidth}
\centering
%\includegraphics[scale=0.4,draft=false]{%
%../../results/motifsym/motifsym_2.pdf}
\caption{$n=100$, $p=0.15$, $q=0.1$}
\end{subfigure}
\caption{ARI violin plots for the symmetric two-block DSBM}
\label{fig:motifsym}
\end{figure}

\subsection{Asymmetric two-block DSBMs} \label{sec:motif_asymm_dsbms}

We define the \emph{asymmetric two-block DSBM} as the DSBM with $k=2$,
$n_1=n_2=n$ and
$F =
\begin{psmallmatrix}
p & q_1 \\ q_2 & p
\end{psmallmatrix}$
where $q_1 > q_2$ and $p = \frac{1}{2}(q_1+q_2)$.
Figure~\ref{fig:asym_two_block_dsbm} shows this model.

\begin{figure}[H]
\centering
%\includegraphics[scale=0.8,draft=false]{%
%../tikz/asym_two_block_dsbm/asym_two_block_dsbm.pdf}
\caption{Asymmetric two-block DSBM block structure and sparsity matrix}
\label{fig:asym_two_block_dsbm}
\end{figure}

We test the performance of Algorithm~\ref{alg:motifrwspectclust} across various
motifs with parameters $k=l=2$ on this model.
Figure~\ref{fig:motifasym} shows violin plots over 20 trials of ARI against
motif, for different sets of parameters $n,p,q_1,q_2$, and $|C|$ is shown.
It is apparent that motif-based clustering with $\ca{M}_1$ is the best method,
consistently achieving the highest ARI and keeping $|C|$ at its maximum value
of $2n$.
It is unsurprising that $\ca{M}_1$ (feed-back loop) performs well on this
model; large $p$ makes feed-back loops within clusters likely, and small $q_2$
makes feed-back loops spanning the clusters unlikely. Motif $\ca{M}_2$ also
performs reasonably well since it contains $\ca{M}_1$ as a submotif.
Furthermore, the constraint $p = \frac{1}{2}(q_1+q_2)$ ensures that the na\"ive
symmetrisation $M=G+G^\top$ produces indistinguishable clusters, and hence the
traditional method performs extremely poorly.

\begin{figure}[H]
\begin{subfigure}{.49\textwidth}
\centering
%\includegraphics[scale=0.4,draft=false]{%
%../../results/motifasym/motifasym_1.pdf}
\caption{$n=100$, $p=0.2$, $q_1=0.35$, $q_2=0.05$}
\end{subfigure}
\begin{subfigure}{.49\textwidth}
\centering
%\includegraphics[scale=0.4,draft=false]{%
%../../results/motifasym/motifasym_2.pdf}
\caption{$n=200$, $p=0.15$, $q_1=0.25$, $q_2=0.05$}
\end{subfigure}
\caption{ARI violin plots for the asymmetric two-block DSBM}
\label{fig:motifasym}
\end{figure}

\section{US Political Blogs network} \label{sec:motif_polblogs}

Our first real data set is the US Political Blogs network
\cite{adamic2005political}, consisting of data collected two months before the
2004 US election. Vertices represent blogs, and are labelled by their political
leaning (`liberal' or `conservative'). Weighted directed edges represent the
number of citations from one blog to another. After preprocessing
(Section~\ref{sec:notes_preprocessing}) there are $536$ liberal blogs, $636$
conservative blogs (total 1222) and $19 \, 024$ edges. The network is plotted
in Figure~\ref{fig:polblogs_network}.

We test the performance of Algorithm~\ref{alg:motifrwspectclust} across various
motifs with parameters $k=l=2$ on this network.
Figure~\ref{fig:polblogs_ariplot} plots ARI against component size $|C|$.
There is an apparent trade-off between ARI and connected component size.
Motif $\ca{M}_9$ clusters many vertices with $|C|=1197$ and an ARI of 0.82,
while the more strongly connected $\ca{M}_4$ only clusters $378$ vertices, with
an improved ARI of 0.92.
Finally, the poor performance of traditional spectral clustering is due to a
small number of very weakly connected vertices being partitioned off, indicated
by the dashed line and circled vertices in Figure~\ref{fig:polblogs_network}.

\vspace*{0.5cm}
\begin{figure}[H]
\begin{subfigure}{.49\textwidth}
\centering
%\includegraphics[scale=0.4,draft=false]{%
%../../results/polblogs/polblogs_network.pdf}
\caption{The US Political Blogs network}
\label{fig:polblogs_network}
\end{subfigure}
\begin{subfigure}{.49\textwidth}
\centering
%\includegraphics[scale=0.4,draft=false]{%
%../../results/polblogs/polblogs_ari_conn.pdf}
\caption{ARI against $|C|$ across motifs}
\label{fig:polblogs_ariplot}
\end{subfigure}
\caption{Plots relating to the US Political Blogs network}
\end{figure}

Figure~\ref{fig:polblogs_embedding} shows the embedding given by eigenvectors 2
and 3 of the random-walk Laplacian of the MAM generated by motif $\ca{M}_{12}$.
An instance of this motif in the network indicates the presence of a pair of
mutually citing blogs with an incoming citation from a third (see
Figure~\ref{fig:motif_definitions_directed}). Colourings are provided for
Figure~\ref{fig:polblogs_embedding_truth} by the truth labels and for
Figure~\ref{fig:polblogs_embedding_kmeans} by the $k$-means++ clustering of
eigenvector 2. The clusterings are very similar, giving an ARI of $0.82$.

\vspace*{0.5cm}
\begin{figure}[H]
\begin{subfigure}{.49\textwidth}
\centering
%\includegraphics[scale=0.4,draft=false]{%
%../../results/polblogs/polblogs_M12_truth.pdf}
\caption{Colouring by truth label}
\label{fig:polblogs_embedding_truth}
\end{subfigure}
\begin{subfigure}{.49\textwidth}
\centering
%\includegraphics[scale=0.4,draft=false]{%
%../../results/polblogs/polblogs_M12_clusts.pdf}
\caption{Colouring by $k$-means++ cluster}
\label{fig:polblogs_embedding_kmeans}
\end{subfigure}
\caption{Eigendecomposition embedding of the US Political Blogs network}
\label{fig:polblogs_embedding}
\end{figure}

\pagebreak

\section{US Migration network} \label{sec:motif_migration}

The next data set is the US Migration network \cite{census2000}, consisting of
data collected during the US Census in 2000. Vertices represent the 3075
counties in 49 contiguous states (excluding Alaska and Hawaii, and including
the District of Columbia). The $721\,432$ weighted directed edges represent the
number of people migrating from county to county, capped at $10 \, 000$ (the
99.9th percentile) to control large entries, as in \cite{DirectedClustImbCuts}.

We test the performance of Algorithm~\ref{alg:motifrwspectclust} with three
selected motifs: $\ca{M}_\mathrm{s}$, $\ca{M}_6$ and $\ca{M}_9$ (see
Figure~\ref{fig:motif_definitions_directed}).
$\ca{M}_\mathrm{s}$ gives the traditional spectral clustering method with
na\"ive symmetrisation.
$\ca{M}_6$ represents a pair of counties exchanging migrants, with both also
receiving migrants from a third.
$\ca{M}_9$ is a path of length two, allowing counties to be deemed similar if
there is migration between them via another.

Firstly, we plot sweep profiles of the graph using the second eigenvector of
the random-walk Laplacian of the MAM associated with each motif, in
Figure~\ref{fig:migration_sweep}. Note that all three display clear minima,
indicating that these motifs produce well-defined clusters. The two-part
clusterings produced by eigenvector sweep are somewhat similar across the three
motifs, with pairwise ARIs equal to $\textrm{ARI}(\ca{M}_\mathrm{s}, \ca{M}_6)
= 0.67$, $\textrm{ARI}(\ca{M}_\mathrm{s}, \ca{M}_9) = 0.92$ and
$\textrm{ARI}(\ca{M}_6, \ca{M}_9) = 0.73$.

\begin{figure}[H]
\begin{subfigure}{.325\textwidth}
\centering
%\includegraphics[scale=0.4,draft=false]{%
%../../results/us_migration/us_migration_sweep_profile_Ms.pdf}
\caption{$\ca{M}_\mathrm{s}$}
\end{subfigure}
\begin{subfigure}{.325\textwidth}
\centering
%\includegraphics[scale=0.4,draft=false]{%
%../../results/us_migration/us_migration_sweep_profile_M6.pdf}
\caption{$\ca{M}_6$}
\end{subfigure}
\begin{subfigure}{.325\textwidth}
\centering
%\includegraphics[scale=0.4,draft=false]{%
%../../results/us_migration/us_migration_sweep_profile_M9.pdf}
\caption{$\ca{M}_9$}
\end{subfigure}
\caption{Sweep profiles of the US Migration network}
\label{fig:migration_sweep}
\end{figure}

Next, Figure~\ref{fig:us_migration} plots maps of the US, with counties
coloured initially by the first six non-trivial eigenvectors $x_2, \ldots, x_7$
of the random-walk Laplacian of the associated MAM, and then by the clustering
$C$ obtained by Algorithm~\ref{alg:motifrwspectclust} with $k=l=7$.

For the eigenvector colourings, note how the coloured regions often line up
with state boundaries, indicating that many migrants stay within the same
state.
It is also apparent that the motifs $\ca{M}_6$ and $\ca{M}_9$ produce `noisier'
embeddings than traditional spectral clustering, due to their reliance on
three-vertex motifs.
Eigenvector~2 approximately differentiates counties by longitude, although
$\ca{M}_9$ achieves a clearer division between east and west, while
$\ca{M}_\mathrm{s}$ and $\ca{M}_6$ colour California (CA, see
Figure~\ref{fig:notes_us_map}) more similarly to the East Coast.
Eigenvector 3 tends to differentiate by latitude, though $\ca{M}_\mathrm{s}$
and $\ca{M}_6$ particularly isolate the states of North Dakota (ND), South
Dakota (SD), Minnesota (MN), Wisconsin (WI) and Michigan (MI).
Further structure is visible across all three motifs for eigenvectors 4--7.

The clusterings $C$ partition the counties into $k=7$ regions, and there are
some interesting differences between the motifs.
Since there is no ground-truth clustering, we record the Ncut score associated
with each clustering.
It is apparent that motifs $\ca{M}_6$ and $\ca{M}_9$ give a similar partition,
although with some differences:
$\ca{M}_6$ clusters the East Coast together with western Florida (FL) and the
counties containing Los Angeles (CA), San Diego (CA), Las Vegas (NV), Phoenix
(AZ), Tucson (AZ), Denver (CO), Chicago (IL) and Nashville (TN).
$\ca{M}_6$ favours a larger `central' region, which includes significant parts
of Colorado (CO), Oklahoma (OK), Arkansas (AR) and Illinois (IL).
$\ca{M}_\mathrm{s}$ gives a somewhat different partition, with one of the
clusters allocated to Michigan (MI) and Wisconsin (WI) rather than Mississippi
(MS), Alabama (AL), Georgia (GA) and Tennessee (TN). As with the eigenvectors,
the clustering is smoother for $\ca{M}_\mathrm{s}$ than for $\ca{M}_6$ and
$\ca{M}_9$.

\pagebreak

\vspace*{-1cm}
\begin{figure}[H]
\begin{table}[H]
\centering
\setlength{\tabcolsep}{0em}
\begin{tabular}{ |c|c|c|c| }
%\expandableinput ../../results/us_migration/us_migration_table.txt
\end{tabular}
\end{table}
\vspace*{-0.5cm}
\caption{Motif-based colourings of the US Migration network}
\label{fig:us_migration}
\end{figure}
\clearpage{}
\clearpage{}
\addtocontents{toc}{\protect\newpage}
\chapter{Bipartite Clustering} \label{chap:bipartite}

We propose a technique for spectral clustering of bipartite graphs and test its
performance on both real and synthetic data.
In Section~\ref{sec:bipartite_graphs} we define bipartite graphs and present
our clustering technique.
In Section~\ref{sec:bipartite_sbms} we propose a bipartite stochastic block
model (BSBM) and perform experiments with varying parameters.
In Section~\ref{sec:bipartite_american_revolution} we demonstrate our method
using the American Revolution network.
In Section~\ref{sec:bipartite_languages} we analyse the Unicode Languages
network.

\section{Bipartite graphs} \label{sec:bipartite_graphs}

\begin{definition}
A \emph{bipartite graph} is a graph $\ca{G}=(\ca{V,E})$ where $\ca{V}$ can be
partitioned into $\ca{V} = \ca{S} \sqcup \ca{D}$ such that $\ca{E} \subseteq
\ca{S} \times \ca{D}$. That is, every edge starts in $\ca{S}$ and ends in
$\ca{D}$. We refer to $\ca{S}$ as the \emph{source vertices} and to $\ca{D}$
as the \emph{destination vertices}.
\end{definition}

\subsection{Collider and expander motifs} \label{sec:coll_expa}

Our method for clustering bipartite graphs revolves around two \emph{anchored}
motifs; the \emph{collider} and the \emph{expander}
(Figure~\ref{fig:expa_coll}). For each motif the anchor set is $\ca{A}=\{ 1,3
\}$.

\begin{figure}[H]
\centering
%\includegraphics[scale=0.8,draft=false]{../tikz/expa_coll/expa_coll.pdf}
\caption{The collider and expander motifs}
\label{fig:expa_coll}
\end{figure}

These motifs are useful for bipartite clustering because of
Proposition~\ref{prop:coll_expa_formulae}, which states that their restricted
MAMs are the adjacency matrices of the
projections~\cite{kolaczyk2014statistical} of the graph $\ca{G}$.
In particular they can be used as similarity matrices for the source and
destination vertices respectively.
The similarity of two distinct source (resp. destination) vertices is the sum
over their mutual neighbours of the average weights of their edges to (resp.
from) that neighbour.

\begin{proposition}[Colliders and expanders in bipartite graphs]
\label{prop:coll_expa_formulae}
Let $\ca{G} = (\ca{V,E},W)$ be a directed bipartite graph. Let
$M_\mathrm{coll}$ and $M_\mathrm{expa}$ be the structural or functional MAMs
of $\ca{M}_\mathrm{coll}$ and $\ca{M}_\mathrm{expa}$ respectively in
$\ca{G}$. Then
%
\begin{align*}
(M_\mathrm{coll})_{i j} &= \bb{I} \{i \neq j\} \hspace*{-0.4cm}
\sum_{\substack{k \in \ca{D} \\ (i,k), (j,k) \in \ca{E}}} \hspace*{-0.2cm}
\frac{1}{2} \Big[ W((i,k)) + W((j,k)) \Big]\,, &(1)\\
(M_\mathrm{expa})_{i j} &= \bb{I} \{i \neq j\} \hspace*{-0.4cm}
\sum_{\substack{k \in \ca{S} \\ (k,i), (k,j) \in \ca{E}}}
\hspace*{-0.2cm}\frac{1}{2} \Big[ W((k,i)) + W((k,j)) \Big]\,. &(2)
\end{align*}
%
\end{proposition}
%
\begin{proof}
See Proof~\ref{proof:coll_expa_formulae}.
\end{proof}

\subsection{Bipartite spectral clustering algorithm}

Algorithm~\ref{alg:bipartite_clustering} gives our procedure for clustering a
bipartite graph. The algorithm uses the collider and expander motifs to create
similarity matrices for the source and destination vertices respectively (as in
Section~\ref{sec:coll_expa}), and then applies random-walk spectral clustering
(Algorithm~\ref{alg:rwspectclust}) to produce the partitions.

\vspace*{0.5cm}
\begin{algorithm}[H]

\SetKwFunction{Main}{BipartiteRWSpectClust}
\newcommand{\MainArgs}{$\ca{G},k_\ca{S},k_\ca{D},l_\ca{S},l_\ca{D}$}

\BlankLine
\Input{Bipartite graph $\ca{G}$, source clusters $k_\ca{S}$, destination
clusters $k_\ca{D}$, source dimension $l_\ca{S}$, destination dimension
$l_\ca{D}$}
\Output{Source partition $\ca{S}_1, \ldots, \ca{S}_{k_\ca{S}}$, destination
partition $\ca{D}_1, \ldots, \ca{D}_{k_\ca{D}}$}
\BlankLine

\Function{\Main{\MainArgs}}{
Construct the collider motif adjacency matrix $M_\mathrm{coll}$ of the
graph $\ca{G}$ \\
Construct the expander motif adjacency matrix $M_\mathrm{expa}$ of the
graph $\ca{G}$ \\
$M_\mathrm{coll} \leftarrow M_\mathrm{coll}[\ca{S,S}]$ \Comm*{restrict rows
and columns of $M_\mathrm{coll}$ to $\ca{S}$ \hspace*{0.07cm}}
$M_\mathrm{expa} \leftarrow M_\mathrm{expa}[\ca{D,D}]$ \Comm*{restrict rows
and columns of $M_\mathrm{expa}$ to $\ca{D}$}
$\ca{S}_1, \ldots, \ca{S}_{k_\ca{S}} \leftarrow$
\texttt{RWSpectClust($M_\mathrm{coll},k_\ca{S},l_\ca{S}$)} \\
$\ca{D}_1, \ldots, \ca{D}_{k_\ca{D}} \leftarrow$
\texttt{RWSpectClust($M_\mathrm{expa},k_\ca{D},l_\ca{D}$)} \\
\Return $\ca{S}_1, \ldots, \ca{S}_{k_\ca{S}}$ and $\ca{D}_1, \ldots,
\ca{D}_{k_\ca{D}}$
}

\caption{Bipartite random walk spectral clustering}
\label{alg:bipartite_clustering}
\end{algorithm}

\section{Bipartite stochastic block models} \label{sec:bipartite_sbms}

We define the \emph{bipartite stochastic block model} (BSBM)
\cite{florescu2016spectral} as the DSBM with $k=4$, $n_1 = \dots = n_4=n$ and
$F =
\begin{psmallmatrix}
0 & 0 & p & q \\ 0 & 0 & q & p \\ 0 & 0 & 0 & 0 \\ 0
& 0 & 0 & 0
\end{psmallmatrix}$
where $p > q$. Figure~\ref{fig:bipartite_bsbm} illustrates the block structure
and sparsity matrix of this model. This model partitions the source vertices as
$\ca{S} = \ca{S}_1 \sqcup \ca{S}_2$ and the destination vertices as
$\ca{D}=\ca{D}_1 \sqcup \ca{D}_2$. Edges exist with high probability from
$\ca{S}_1$ to $\ca{D}_1$ and from $\ca{S}_2$ to $\ca{D}_2$.

\begin{figure}[H]
\centering
%\includegraphics[scale=0.8,draft=false]{%
%../tikz/bipartite_dsbm/bipartite_dsbm.pdf}
\caption{BSBM block structure and sparsity matrix}
\label{fig:bipartite_bsbm}
\end{figure}

We test the performance of Algorithm~\ref{alg:bipartite_clustering} with
parameters $k_\ca{S} = k_\ca{D} = l_\ca{S} = l_\ca{D} = 2$ on this model.
For comparison we implement the co-clustering method from \cite{dhillon2001co},
which is based on random-walk spectral clustering of the symmetrised adjacency
matrix $G+G^\top$.
Figure~\ref{fig:bipartite} shows violin plots over 20 trials of ARI against
method, for different sets of parameters $n,p,q$.
Note that if a bipartite graph is connected, then so are $M_\mathrm{coll}$ and
$M_\mathrm{expa}$, so we need not consider the largest connected component size
$|C|$.
Performance of the two methods is very similar, for source and destination
vertices.

\begin{figure}[H]
\begin{subfigure}{.49\textwidth}
\centering
%\includegraphics[scale=0.4,draft=false]{%
%../../results/bipartite/bipartite1.pdf}
\caption{$n=100$, $p=0.2$, $q=0.1$}
\end{subfigure}
\begin{subfigure}{.49\textwidth}
\centering
%\includegraphics[scale=0.4,draft=false]{%
%../../results/bipartite/bipartite2.pdf}
\caption{$n=200$, $p=0.1$, $q=0.06$}
\end{subfigure}
\caption{ARI violin plots for the BSBM}
\label{fig:bipartite}
\end{figure}

\section{American Revolution network} \label{sec:bipartite_american_revolution}

As an example of application of our bipartite clustering method to real data,
we consider the American Revolution network \cite{konect:brunson_revolution}.
This consists of data collected from before the American Revolution. Source
vertices are people, and destination vertices are organisations. Edges
represent membership of a person to an organisation. There are 136 people, 5
organisations and 160 edges.

Algorithm~\ref{alg:bipartite_clustering} is run on the American Revolution
network, with parameters $k_\ca{S} = l_\ca{S} = 5$ and $k_\ca{D} = l_\ca{D} =
2$. Figure~\ref{fig:bipartite_revolution_source} plots the network with people
coloured by source cluster, and Figure~\ref{fig:bipartite_revolution_dest}
plots the network with organisations coloured by destination cluster. The
algorithm succeeds in clustering people based on their common memberships, and
in clustering organisations based on their common members.

\begin{figure}[H]
\begin{subfigure}{.49\textwidth}
\centering
%\includegraphics[scale=0.4,draft=false]{%
%../../results/american_revolution/american_revolution_source.pdf}
\caption{Grouping people into 5 clusters}
\label{fig:bipartite_revolution_source}
\end{subfigure}
\begin{subfigure}{.49\textwidth}
\centering
%\includegraphics[scale=0.4,draft=false]{%
%../../results/american_revolution/american_revolution_dest.pdf}
\caption{Grouping organisations into 2 clusters}
\label{fig:bipartite_revolution_dest}
\end{subfigure}
\caption{Bipartite clustering of the American Revolution network}
\label{fig:bipartite_revolution}
\end{figure}

\section{Unicode Languages network} \label{sec:bipartite_languages}

The final data set is the Unicode Languages network \cite{konect:unicodelang},
consisting of data collected in 2014 on languages spoken around the world.
Source vertices are territories, and destination vertices are languages.
Weighted directed edges from territory to language indicate the number of
inhabitants in that territory who speak the specified language (territory
population data taken from \cite{geonames}).
After preprocessing (Section~\ref{sec:notes_preprocessing}) there are $155$
territories, $270$ languages and $705$ edges.

We test Algorithm~\ref{alg:bipartite_clustering} with parameters $k_\ca{S} =
l_\ca{S} = k_\ca{D} = l_\ca{D} = 6$ on this network.
For the source vertices, Figure~\ref{fig:bipartite_languages_map} plots maps of
the world with territories coloured by the clustering obtained. The top 20
territories (by population) in each cluster are given in
Table~\ref{tab:bipartite_languages_source_clusters}.
Cluster~1 is by far the largest cluster, and includes a wide variety of
territories, of which many but not all speak some English.
Cluster~2 contains the Persian-speaking territories of Iran and Afghanistan,
the Arabic territories of Saudi Arabia and Syria, and the African
French-speaking DR Congo, C\^ote d'Ivoire, Burkina Faso, Niger and others. It
also includes Haiti, another French-speaking territory.
Cluster~3 mostly captures Spanish-speaking territories in the Americas and also
contains Equatorial Guinea, another Spanish-speaking territory in Africa.
Cluster~4 includes the Slavic territories of Russia and some of its neighbours.
The absence of Kazakhstan may be due to the $981 \, 760$ Kazakhs who speak
German which is not a Slavic or Turkic language.
Cluster~5 covers China, Hong Kong, Mongolia and some of South-East Asia. The
inclusion of Panama might be due to the $6821$ Panamanians who speak Chinese.
Cluster~6 is the smallest cluster and contains only Japan and the Koreas, which
are connected by the $636 \, 440$ Japanese who speak Korean.

There are a few territories and languages which are not contained in the large
connected component of the network due to their linguistic isolation. These
territories are Laos, Norway and Timor-Leste, and the languages are Lao,
Norwegian Bokm{\aa}l and Norwegian Nynorsk.

\begin{figure}[H]
\centering
%\includegraphics[scale=0.6, draft=false]{%
%../../results/languages/languages_source_map_clusts.pdf}
\caption{Clustering the territories from the Unicode Languages network}
\label{fig:bipartite_languages_map}
\end{figure}

\begin{table}[H]
\centering
\scriptsize
\begin{tabular}{ |c|c|c|c|c|c| }
\hline
\rule{0pt}{1.2em}
\cellcolor[HTML]{8DD3C7} Cluster 1 &
\cellcolor[HTML]{FFFFB3} Cluster 2 &
\cellcolor[HTML]{BEBADA} Cluster 3 &
\cellcolor[HTML]{FB8072} Cluster 4 &
\cellcolor[HTML]{80B1D3} Cluster 5 &
\cellcolor[HTML]{FDB462} Cluster 6 \\[0.1cm]
\hline \rule{0pt}{1.2em}
India & Iran & Mexico & Russia & China & Japan \\
United States & DR Congo & Colombia & Ukraine & Indonesia & S.\ Korea \\
Brazil & Afghanistan & Argentina & Uzbekistan & Vietnam & N.\ Korea \\
Pakistan & Saudi Arabia & Peru & Belarus & Malaysia & \\
Bangladesh & Syria & Venezuela & Tajikistan & Taiwan & \\
Nigeria & C\^ote d'Ivoire & Ecuador & Kyrgyzstan & Cambodia & \\
Philippines & Burkina Faso & Guatemala & Turkmenistan & Hong Kong & \\
Ethiopia & Niger & Cuba & Georgia & Singapore & \\
Germany & Mali & Bolivia & Moldova & Panama & \\
Egypt & Senegal & Paraguay & Latvia & Mongolia & \\
Turkey & Tunisia & El Salvador & Estonia & & \\
Thailand & Chad & Nicaragua & & & \\
France & Guinea & Costa Rica & & & \\
United Kingdom & Somalia & Uruguay & & & \\
Italy & Burundi & Eq.\ Guinea & & & \\
Myanmar & Haiti & & & & \\
South Africa & Benin & & & & \\
Spain & Azerbaijan & & & & \\
Tanzania & Togo & & & & \\
Kenya & Libya & & & & \\
$\cdots$ & $\cdots$ & & & & \\
$|\textrm{Cluster\ } 1 |$ = 87 &
$|\textrm{Cluster\ } 2 |$ = 29 &
$|\textrm{Cluster\ } 3 |$ = 15 &
$|\textrm{Cluster\ } 4 |$ = 11 &
$|\textrm{Cluster\ } 5 |$ = 10 &
$|\textrm{Cluster\ } 6 |$ = 3
\\[0.1cm]
\hline
\end{tabular}
\caption{Clustering the territories from the Unicode Languages network}
\label{tab:bipartite_languages_source_clusters}
\end{table}

For the destination vertices, we present the six clusters obtained by
Algorithm~\ref{alg:bipartite_clustering}.
Table~\ref{tab:bipartite_languages_dest_clusters} contains the top 20 languages
(by number of speakers) in each cluster.
Cluster~1 is the largest cluster and contains the European languages of
Spanish, Portuguese and French, as well as dialects of Arabic.
Cluster~2 is also large and includes English as well as several South Asian
languages such as Hindi, Bengali, Urdu and Punjabi.
Cluster~3 consists of many indigenous African languages such as Swahili,
Kinyarwanda and Somali.
Cluster~4 captures languages from South-East Asia, mostly spoken in Indonesia
and Malaysia.
Cluster~5 identifies several varieties of Chinese and a few other Central and
East Asian languages such as Kazakh and Uighur. Interestingly Korean is also
placed in this group and not with Japanese, even though the Koreas are
clustered together with Japan in
Table~\ref{tab:bipartite_languages_source_clusters}.
Cluster~6 captures more South-East Asian languages, this time from Thailand,
Myanmar and Cambodia. Pattani Malay is in this cluster because despite its name
it is spoken more in Thailand than in Malaysia.

\vspace*{0.5cm}
\begin{table}[H]
\centering
\scriptsize
\begin{tabular}{ |c|c|c|c|c|c| }
\hline
\rule{0pt}{1.2em}
Cluster 1 & Cluster 2 & Cluster 3 & Cluster 4 &
Cluster 5 & Cluster 6 \\[0.1cm]
\hline \rule{0pt}{1.2em}
Spanish & English & Swahili & Indonesian & Chinese & Thai \\
Arabic & Hindi & Kinyarwanda & Javanese & Wu Chinese & N.E.\ Thai \\
Portuguese & Bengali & Somali & Malay & Korean & Khmer \\
French & Urdu & Luba-Lulua & Sundanese & Xiang Chinese & N.\ Thai \\
Russian & Punjabi & Kikuyu & Madurese & Hakka Chinese & S.\ Thai \\
Japanese & Telugu & Congo Swahili & Minangkabau & Minnan Chinese & Shan \\
German & Marathi & Luyia & Betawi & Gan Chinese & Pattani Malay \\
Turkish & Vietnamese & Ganda & Balinese & Kazakh & \\
Persian & Tamil & Luo & Buginese & Uighur & \\
Italian & Lahnda & Sukuma & Banjar & Sichuan Yi & \\
Egyptian Arabic & Filipino & Kalenjin & Achinese & Mongolian & \\
Polish & Gujarati & Lingala & Sasak & Zhuang & \\
Nigerian Pidgin & Kannada & Nyankole & Makasar & Tibetan & \\
Ukrainian & Pushto & Gusii & Lampung Api & & \\
Dutch & Malayalam & Kiga & Rejang & & \\
Algerian Arabic & Oriya & Soga & & & \\
Moroccan Arabic & Burmese & Luba-Katanga & & & \\
Hausa & Bhojpuri & Meru & & & \\
Azerbaijani & Amharic & Teso & & & \\
Uzbek & Oromo & Nyamwezi & & & \\
$\cdots$ & $\cdots$ & $\cdots$ & & & \\
$|\textrm{Cluster\ } 1 |$ = 120 &
$|\textrm{Cluster\ } 2 |$ = 90 &
$|\textrm{Cluster\ } 3 |$ = 25 &
$|\textrm{Cluster\ } 4 |$ = 15 &
$|\textrm{Cluster\ } 5 |$ = 13 &
$|\textrm{Cluster\ } 6 |$ = 7
\\[0.1cm]
\hline
\end{tabular}
\caption{Clustering the languages from the Unicode Languages network}
\label{tab:bipartite_languages_dest_clusters}
\end{table}

\clearpage{}
\clearpage{}
\chapter{Conclusion} \label{chap:conclusions}

With this dissertation we have introduced a graph-theoretic framework for
analysis of weighted directed networks, and presented new matrix-based formulae
for MAMs (Chapter~\ref{chap:graphs}).
We have summarised the method of random-walk spectral clustering and shown how
it can be used with motif-based techniques (Chapter~\ref{chap:spectral}).
We have presented results from the application of a motif-based method both to
synthetic data (DSBMs) and to real data (US Political Blogs network, US
Migration network). We have demonstrated that this technique outperforms
traditional spectral clustering methods on several occasions
(Chapter~\ref{chap:motif}).
We have introduced a motif-based spectral method for clustering bipartite
graphs and presented results both from synthetic data (BSBMs) and from real
data (American Revolution network, Unicode Languages network).

In particular we have shown that motif-based spectral clustering is a valuable
tool for clustering weighted directed networks, which is scalable and easy to
implement.
Superior performance has been demonstrated especially with asymmetric DSBMs in
Section~\ref{sec:motif_asymm_dsbms}, and with the US Political Blogs network in
Section~\ref{sec:motif_polblogs}.

\section*{Limitations}

There are limitations to our work.
While our matrix-based formulae for MAMs are simple to implement and moderately
scalable, they are computationally unwieldy for large networks (see
Section~\ref{sec:notes_computation} for details).
As mentioned in~\cite{benson2016higher}, fast triangle enumeration
algorithms~\cite{demeyer2013ISMA,wernicke2006efficient,wernicke2006fanmod}
offer increased performance, at the expense of methodological simplicity.
Another shortcoming of the matrix-based formulae is that unlike motif detection
algorithms such as~\cite{wernicke2006fanmod}, they do not extend to motifs on
four or more vertices.

\section*{Future work}

There is plenty of scope for methodological investigation related to our work.
Simple extensions could involve an analysis of the differences between
clustering methods based on functional and structural MAMs respectively.
One could also experiment with the effects of replacing the random-walk
Laplacian with the unnormalised Laplacian or symmetric normalised Laplacian
\cite{von2007tutorial}.
Similarly one might try replacing Ncut with RatioCut \cite{hagen1992new}. We
note that although our methods apply to weighted graphs, we have only discussed
unweighted DSBMs. Therefore it would be interesting to investigate weighted
DSBMs (perhaps following the exponential family method detailed in
\cite{aicher2013adapting}) and to use them for evaluation of motif-based
spectral clustering procedures.

Further experimental work is also desirable. We would like to conduct
experiments on more real data, and suggest that collaboration networks such
as~\cite{snap:astro}, and bipartite preference networks such
as~\cite{icon:movie} could be interesting.
Comparison with other clustering methods could also be insightful; the
Hermitian matrices method in~\cite{DirectedClustImbCuts}, the PageRank method
in~\cite{yin2017local} and \textsc{Tectonic}
from~\cite{tsourakakis2017scalable} may give suitable benchmarks for
performance.

\clearpage{}
%TC:ignore

% enable appendix numbering format and include appendices
\appendix
\fancyhead[RO]{\itshape{\nouppercase{Appendix \thechapter : \leftmark}}}

%TC:endignore
\clearpage{}
\chapter{Proofs and Examples}\label{chap:appendix_proofs}

\section{Proofs}

\begin{prf}[Proposition~\ref{prop:motif_adj_matrix_formula}, MAM formula]
\label{proof:motif_adj_matrix_formula}
%
Consider $(1)$. We sum over functional instances $\ca{M} \cong \ca{H} \leq
\ca{G}$ such that $\{i,j\} \in \ca{A(H)}$.
This is equivalent to summing over $\{k_2, \ldots, k_{m-1}\} \subseteq
\ca{V}$ and $\sigma \in S_\ca{M,A}^\sim$, such that $k_u$ are all distinct
and
%
$$ (u,v) \in \ca{E_M} \implies (k_{\sigma u}, k_{\sigma v}) \in \ca{E}\,.
\qquad (\dagger) $$
%
This is because the vertex set $\{k_2, \ldots, k_{m-1}\} \subseteq \ca{V}$
indicates which vertices are present in the instance $\ca{H}$, and $\sigma$
describes the mapping from $\ca{V_M}$ onto those vertices: $u \mapsto
k_{\sigma u}$. We take $\sigma \in S_\ca{M,A}^\sim$ to ensure that $\{i,j\}
\in \ca{A(H)}$ (since $i=k_1, \ j=k_m$), and that instances are counted
exactly once.
The condition $(\dagger)$ is to check that $\ca{H}$ is a functional instance
of $\ca{M}$ in $\ca{G}$. Hence
%
\begin{align*}
M^\mathrm{func}_{i j} &= \frac{1}{|\ca{E_M}|}
\sum_{\ca{M} \cong \ca{H} \leq
\ca{G}} \bb{I} \big\{ \{i,j\} \in \ca{A}(\ca{H}) \big\} \sum_{e \in
\ca{E_H}} W(e) \\
%
&=  \frac{1}{|\ca{E_M}|} \sum_{\{ k_2, \ldots, k_{m-1} \}} \sum_{\sigma \in
S_\ca{M,A}^\sim} \bb{I} \big\{ k_u \textrm{ all distinct}, \, (\dagger)
\big\} \sum_{e \in \ca{E_H}} W(e)\,.
\end{align*}
%
For the first term, by conditioning on the types of edge in $\ca{E_M}$:
\begin{align*}
%
\bb{I} \big\{ k_u \textrm{ all distinct}, \, (\dagger) \big\}
&= \prod_{\ca{E}_\ca{M}^0} \bb{I} \{ k_{\sigma u} \neq k_{\sigma v} \} \\
& \qquad \times \prod_{\ca{E}_\ca{M}^\mathrm{s}} \bb{I} \{ (k_{\sigma u},
k_{\sigma v}) \in \ca{E} \} \\
& \qquad \times \prod_{\ca{E}_\ca{M}^\mathrm{d}} \bb{I} \{(k_{\sigma u},
k_{\sigma v}) \in \ca{E} \textrm{ and } (k_{\sigma v}, k_{\sigma u}) \in
\ca{E}\} \\
%
&= \prod_{\ca{E}_\ca{M}^0} (J_\mathrm{n})_{k_{\sigma u},k_{\sigma v}}
\prod_{\ca{E}_\ca{M}^\mathrm{s}} J_{k_{\sigma u},k_{\sigma v}}
\prod_{\ca{E}_\ca{M}^\mathrm{d}} (J_\mathrm{d})_{k_{\sigma u},k_{\sigma v}}
\\
%
&= J^\mathrm{func}_{\mathbf{k},\sigma}\,.
%
\end{align*}
%
Assuming $\big\{ k_u \textrm{ all distinct}, \, (\dagger) \big\}$, the second
term is
%
\begin{align*}
%
\sum_{e \in \ca{E_H}} W(e)
&= \sum_{\ca{E}_\ca{M}^\mathrm{s}} W((k_{\sigma u},k_{\sigma v}))
+ \sum_{\ca{E}_\ca{M}^\mathrm{d}} \big( W((k_{\sigma u},k_{\sigma v})) +
W((k_{\sigma v},k_{\sigma u})) \big) \\
%
&= \sum_{\ca{E}_\ca{M}^\mathrm{s}} G_{k_{\sigma u},k_{\sigma v}}
+ \sum_{\ca{E}_\ca{M}^\mathrm{d}} (G_\mathrm{d})_{k_{\sigma u},k_{\sigma
v}} \\
%
&= G^\mathrm{func}_{\mathbf{k},\sigma}
\end{align*}
%
as required. For $(2)$, we simply change $(\dagger)$ to $(\ddagger)$ to check
that an instance is a \emph{structural} instance:
%
$$ (u,v) \in \ca{E_M} \iff (k_{\sigma u}, k_{\sigma v}) \in \ca{E} \qquad
(\ddagger) $$
%
Now for the first term:
%
\begin{align*}
%
\bb{I} \big\{ k_u \textrm{ all distinct}, \, (\ddagger) \big\}
&= \prod_{\ca{E}_\ca{M}^0} \bb{I} \{(k_{\sigma u}, k_{\sigma v}) \notin
\ca{E} \textrm{ and } (k_{\sigma v}, k_{\sigma u}) \notin \ca{E}\} \\
& \qquad \times \prod_{\ca{E}_\ca{M}^\mathrm{s}} \bb{I} \{(k_{\sigma u},
k_{\sigma v}) \in \ca{E} \textrm{ and } (k_{\sigma v}, k_{\sigma u})
\notin
\ca{E}\} \\
& \qquad \times \prod_{\ca{E}_\ca{M}^\mathrm{d}} \bb{I} \{(k_{\sigma u},
k_{\sigma v}) \in \ca{E} \textrm{ and } (k_{\sigma v}, k_{\sigma u}) \in
\ca{E}\} \\
%
&= \prod_{\ca{E}_\ca{M}^0} (J_\mathrm{0})_{k_{\sigma u},k_{\sigma v}}
\prod_{\ca{E}_\ca{M}^\mathrm{s}} (J_\mathrm{s})_{k_{\sigma u},k_{\sigma v}}
\prod_{\ca{E}_\ca{M}^\mathrm{d}} (J_\mathrm{d})_{k_{\sigma u},k_{\sigma v}}
\\
%
&= J^\mathrm{struc}_{\mathbf{k},\sigma}\,.
%
\end{align*}
%
Assuming $\big\{ k_u \textrm{ all distinct}, \, (\ddagger) \big\}$, the
second term is
%
\begin{align*}
%
\sum_{e \in \ca{E_H}} W(e)
&= \sum_{\ca{E}_\ca{M}^\mathrm{s}} W((k_{\sigma u},k_{\sigma v}))
+ \sum_{\ca{E}_\ca{M}^\mathrm{d}} \big( W((k_{\sigma u},k_{\sigma v})) +
W((k_{\sigma v},k_{\sigma u})) \big) \\
%
&= \sum_{\ca{E}_\ca{M}^\mathrm{s}} (G_\mathrm{s})_{k_{\sigma u},k_{\sigma
v}}
+ \sum_{\ca{E}_\ca{M}^\mathrm{d}} (G_\mathrm{d})_{k_{\sigma u},k_{\sigma
v}} \\
%
&= G^\mathrm{struc}_{\mathbf{k},\sigma}\,.
\end{align*}

\hfill $\square$
\end{prf}

\pagebreak

\begin{prf}[Proposition~\ref{prop:motif_adj_matrix_computation},
Complexity of MAM formula]
\label{proof:motif_adj_matrix_computation}
Suppose ${m \leq 3}$ and consider $M^\mathrm{func}$. The adjacency and
indicator matrices of $\ca{G}$ are
%
\begin{equation*}
\begin{aligned}[c]
&(1) \quad J = \bb{I} \{ G>0 \}\,, \\
&(2) \quad J_0 = \bb{I} \{ G + G^\top = 0 \} \circ J_\mathrm{n}\,, \\
&(3) \quad J_\mathrm{s} = J - J_\mathrm{d}\,, \\
&(4) \quad G_\mathrm{d} = (G + G^\top) \circ J_\mathrm{d} \,,
\end{aligned}
\hspace*{2cm}
\begin{aligned}[c]
&(5) \quad J_\mathrm{n} = \bb{I} \{I_{n \times n} = 0 \}\,, \\
&(6) \quad J_\mathrm{d} = J \circ J^\top\,, \\
&(7) \quad G_\mathrm{s} = G \circ J_\mathrm{s}\,, \\
&
\end{aligned}
\end{equation*}
%
and are computed using four additions and four element-wise multiplications.
$J^\mathrm{func}_{\mathbf{k},\sigma}$ is a product of at most three factors,
and $G^\mathrm{func}_{\mathbf{k},\sigma}$ contains at most three summands, so
%
$$ \sum_{k_2 \in \ca{V}} J^\mathrm{func}_{\mathbf{k},\sigma} \
G^\mathrm{func}_{\mathbf{k},\sigma} $$
%
is expressible as a sum of at most three matrices, each of which is
constructed with at most one matrix multiplication (where $\{k_{\sigma
r},k_{\sigma s}\} \neq \{i,j\}$) and one entry-wise multiplication (where
$\{k_{\sigma r},k_{\sigma s}\} = \{i,j\}$). This is repeated for each $\sigma
\in S_\ca{M,A}^\sim$ (at most six times) and the results are summed.
Calculations are identical for $M^\mathrm{struc}$.

\hfill $\square$
\end{prf}

\begin{prf}[Proposition~\ref{prop:coll_expa_formulae},
Colliders and expanders in bipartite graphs]
\label{proof:coll_expa_formulae}
%
Consider (1) and the collider motif $\ca{M}_\mathrm{coll}$. Since $\ca{G}$ is
bipartite, $M_\mathrm{coll}^\mathrm{func} = M_\mathrm{coll}^\mathrm{struc} =
\vcentcolon M_\mathrm{coll}$, and by Table~\ref{tab:motif_adj_mat_table},
$M_\mathrm{coll} = \frac{1}{2} J_\mathrm{n} \circ (J G^\top + G J^\top)$.
Hence
%
\begin{align*}
(M_\mathrm{coll})_{i j} &= \frac{1}{2} (J_\mathrm{n})_{i j} \ (J G^\top + G
J^\top)_{i j} \\
&= \bb{I}\{i \neq j\} \sum_{k \in \ca{V}} \ \frac{1}{2} \Big(J_{i k} G_{j k}
+ G_{i k} J_{j k} \Big) \\
&= \bb{I}\{i \neq j\} \sum_{k \in \ca{V}} \ \frac{1}{2} \,\bb{I} \, \Big\{
(i,k),(j,k) \in \ca{E} \Big\} \Big[W((i,k)) + W((j,k))\Big] \\
&= \bb{I} \{i \neq j\} \hspace*{-0.4cm} \sum_{\substack{k \in \ca{D} \\
(i,k), (j,k) \in \ca{E}}} \hspace*{-0.2cm} \frac{1}{2} \Big[ W((i,k)) +
W((j,k)) \Big]\,.
\end{align*}
%
Similarly for the expander motif, $M_\mathrm{expa} = \frac{1}{2} J_\mathrm{n}
\circ (J^\top G + G^\top J)$ so
%
\begin{align*}
(M_\mathrm{expa})_{i j} &= \frac{1}{2} (J_\mathrm{n})_{i j} \ (J^\top G +
G^\top J)_{i j} \\
&= \bb{I} \{i \neq j\} \hspace*{-0.4cm} \sum_{\substack{k \in \ca{S} \\
(k,i), (k,j) \in \ca{E}}} \hspace*{-0.2cm} \frac{1}{2} \Big[ W((k,i)) +
W((k,j)) \Big]\,.
\end{align*}
%
\hfill $\square$
\end{prf}

\section{Examples}

\begin{example}[Functional and structural instances]
\label{ex:instances}
Let $\ca{G}=(\ca{V,E})$ be the graph with $\ca{V} = \{ 1,2,3,4 \}$ and
$\ca{E} = \{ (1,2),(1,3),(1,4),(2,3),(3,4),(4,3) \}$. Let $(\ca{M,A})$ be the
anchored motif with $\ca{V_M} = \{1,2,3\}$, $\ca{E_M} =
\{(1,2),(1,3),(2,3)\}$ and $\ca{A} = \{1,3\}$ as defined in Figure
\ref{fig:instance_example_1}.
%
\begin{figure}[H]
\centering
%\includegraphics[scale=0.7,draft=false]{%
%../tikz/instance_example_1/instance_example_1.pdf}
\caption{The specified graph $\ca{G}$ and anchored motif $\ca{M}$}
\label{fig:instance_example_1}
\end{figure}
%
There are three functional instances of $\ca{M}$ in $\ca{G}$, shown in
Figure~\ref{fig:instance_example_2}. However there is just one structural
instance of $\ca{M}$ in $\ca{G}$, given by $\ca{H}_1$. This is because the
double edge $3 \leftrightarrow 4$ in $\ca{G}$ prevents the subgraphs on
$\{1,3,4\}$ from being induced subgraphs.
%
\begin{align*}
\ca{H}_1 &: \quad \ca{V}_1 = \{ 1,2,3 \} ; \quad \ca{E}_1 = \{ (1,2) ,
(2,3) , (1,3) \} ; \quad \ca{A(H}_1) =  \big\{\{1,3\}\big\}\,, \\
\ca{H}_2 &: \quad \ca{V}_2 = \{ 1,3,4 \} ; \quad \ca{E}_2 = \{ (1,3) ,
(1,4) , (3,4) \} ; \quad \ca{A(H}_2) =  \big\{\{1,4\}\big\}\,, \\
\ca{H}_3 &: \quad \ca{V}_3 = \{ 1,3,4 \} ; \quad \ca{E}_3 = \{ (1,3) ,
(1,4) , (4,3) \} ; \quad \ca{A(H}_3) =  \big\{\{1,3\}\big\}\,.
\end{align*}
%
\begin{figure}[H]
\centering
%\includegraphics[scale=0.7,draft=false]{%
%../tikz/instance_example_2/instance_example_2.pdf}
\caption{Functional instances $\ca{H}_1,\ca{H}_2$ and $\ca{H}_3$}
\label{fig:instance_example_2}
\end{figure}

\end{example}

\begin{example}[Motif adjacency matrices]
\label{ex:motif_adj_matrices}
Let $\ca{G}$ and $\ca{(M,A)}$ be as in Example~\ref{ex:instances}, and
suppose $\ca{G}$ has weight map $W((i,j)) \vcentcolon = i + j$. Then using
Definition~\ref{def:motif_adj_matrices} directly, the functional and
structural MAMs of $\ca{(M,A)}$ in $\ca{G}$ are respectively

\vspace*{0.2cm}
$$ %
M^\mathrm{func} =
\begin{pmatrix}
0  & 0  & 28 & 16 \\
0  & 0  & 0  & 0  \\
28 & 0  & 0  & 0  \\
16 & 0  & 0  & 0
\end{pmatrix} \,,
\qquad
M^\mathrm{struc} =
\begin{pmatrix}
0  & 0  & 12 & 0  \\
0  & 0  & 0  & 0  \\
12 & 0  & 0  & 0  \\
0  & 0  & 0  & 0
\end{pmatrix}\,.
$$
\end{example}

\pagebreak

\begin{example}[Calculating an explicit formula for an MAM]
\label{ex:motif_adj_calc}
Consider the functional MAM of the simple motif $\ca{M}_6$
(Figure~\ref{fig:M6}).
%
\begin{figure}[H]
\centering
%\includegraphics[scale=0.7,draft=false]{../tikz/M6/M6.pdf}
\caption{The motif $\ca{M}_6$}
\label{fig:M6}
\end{figure}
%
We use Equation (1) in Proposition~\ref{prop:motif_adj_matrix_formula}.
Firstly, $m = |\ca{V_M}| = 3$ and $|\ca{E_M}| = 4$. The automorphism group of
$\ca{M}_6$ has order 2, corresponding to swapping vertices 1 and 3. Hence
$|S_\ca{M,A}^\sim| = |S_m| / 2 = 6/2 = 3$, and suitable representatives from
$S_\ca{M,A}^\sim$ are

$$ S_\ca{M,A}^\sim = \left\{
%
\sigma_1 =
\begin{pmatrix}
1 & 2 & 3 \\
1 & 2 & 3
\end{pmatrix},
%
\sigma_2 =
\begin{pmatrix}
1 & 2 & 3 \\
2 & 1 & 3
\end{pmatrix},
%
\sigma_3 =
\begin{pmatrix}
1 & 2 & 3 \\
1 & 3 & 2
\end{pmatrix}
\right\}\,. \vspace*{0.2cm}$$
%
So by Proposition~\ref{prop:motif_adj_matrix_formula}, with $i=k_1$ and
$j=k_3$, and writing $k$ for $k_2$:

$$
M^\mathrm{func}_{i j} = \frac{1}{4} \sum_{\sigma \in S_\ca{M,A}^\sim} \
\sum_{k \in \ca{V}} J^\mathrm{func}_{\mathbf{k},\sigma} \
G^\mathrm{func}_{\mathbf{k},\sigma}
$$
%
where since there are no missing edges in $\ca{M}_6$:
%
\begin{align*}
%
J^\mathrm{func}_{\mathbf{k},\sigma}
&= \prod_{\ca{E}_\ca{M}^\mathrm{s}} J_{k_{\sigma u},k_{\sigma v}}
\prod_{\ca{E}_\ca{M}^\mathrm{d}} (J_\mathrm{d})_{k_{\sigma u},k_{\sigma
v}}\,, \\
%
G^\mathrm{func}_{\mathbf{k},\sigma}
&= \sum_{\ca{E}_\ca{M}^\mathrm{s}} G_{k_{\sigma u},k_{\sigma v}}
+ \sum_{\ca{E}_\ca{M}^\mathrm{d}} (G_\mathrm{d})_{k_{\sigma u},k_{\sigma
v}}\,.
%
\end{align*}
%
Writing out the sum over $\sigma$:
%
\begingroup
\allowdisplaybreaks
\begin{align*}
M^\mathrm{func}_{i j}
&= \frac{1}{4} \sum_{k=1}^n J^\mathrm{func}_{\mathbf{k},\sigma_1} \
G^\mathrm{func}_{\mathbf{k},\sigma_1} + \frac{1}{4} \sum_{k=1}^n
J^\mathrm{func}_{\mathbf{k},\sigma_2} \
G^\mathrm{func}_{\mathbf{k},\sigma_2} + \frac{1}{4} \sum_{k=1}^n
J^\mathrm{func}_{\mathbf{k},\sigma_3} \
G^\mathrm{func}_{\mathbf{k},\sigma_3} \\
%
&=         \frac{1}{4} \sum_{k=1}^n J_{j i} J_{j k} (J_\mathrm{d})_{i k}
\big(G_{j i} + G_{j k} + (G_\mathrm{d})_{i k}\big) \\
& \qquad + \frac{1}{4} \sum_{k=1}^n J_{i j} J_{i k} (J_\mathrm{d})_{j k}
\big(G_{i j} + G_{i k} + (G_\mathrm{d})_{j k}\big) \\
& \qquad + \frac{1}{4} \sum_{k=1}^n J_{k i} J_{k j} (J_\mathrm{d})_{i j}
\big(G_{k i} + G_{k j} + (G_\mathrm{d})_{i j}\big) \\
%
& \\
& \\
& \\
&=         \frac{1}{4} J^\top_{i j} \sum_{k=1}^n (J_\mathrm{d})_{i k}
J^\top_{k j} \big(G^\top_{i j} + (G_\mathrm{d})_{i k} + G^\top_{k j}\big) \\
& \qquad + \frac{1}{4} J_{i j} \sum_{k=1}^n J_{i k}
(J_\mathrm{d})_{k j} \big(G_{i j} + G_{i k} + (G_\mathrm{d})_{k j}\big) \\
& \qquad + \frac{1}{4} (J_\mathrm{d})_{i j}
\sum_{k=1}^n J^\top_{i k} J_{k j}
\big((G_\mathrm{d})_{i j} + G^\top_{i k} + G_{k j}\big) \,,
\end{align*}
\endgroup
%
and writing this as a sum of entry-wise and matrix products:
%
\begin{align*}
M^\textrm{func} &= \frac{1}{4} \Big[ J^\top \circ (J_\mathrm{d} G^\top) +
J^\top \circ (G_\mathrm{d} J^\top) + G^\top \circ (J_\mathrm{d} J^\top)
\Big] \\
& \qquad + \frac{1}{4} \Big[ J \circ (J G_\mathrm{d}) + J \circ (G
J_\mathrm{d}) + G \circ (J J_\mathrm{d}) \Big] \\
& \qquad + \frac{1}{4} \Big[ J_\mathrm{d} \circ (J^\top G) + J_\mathrm{d}
\circ (G^\top J) + G_\mathrm{d} \circ (J^\top J) \Big]
\end{align*}
%
where $A \circ B$ is an entry-wise product and $AB$ is a matrix product.
Finally, setting
$$C = J \circ (J G_\mathrm{d}) + J \circ (G J_\mathrm{d}) + G \circ (J
J_\mathrm{d}) + J_\mathrm{d} \circ (J^\top G)\,, $$
and
$$ C' = G_\mathrm{d} \circ (J^\top J)\,, $$
then we have that
$$ M^\mathrm{func} = \frac{1}{4} \big(C + C^\top + C' \big)\,. $$
as in Table~\ref{tab:motif_adj_mat_table}, achieved with just five matrix
multiplications, nine entry-wise multiplications and nine matrix additions
(including the four entry-wise multiplications and four additions needed to
construct the adjacency and indicator matrices).
\end{example}
\clearpage{}
\clearpage{}
\chapter{Motif Adjacency Matrix Formulae}
\label{chap:appendix_matrices}

We give explicit matrix-based formulae for functional motif adjacency matrices
$M^\mathrm{func}$ for all simple motifs $\ca{M}$ on at most three vertices,
along with the anchored motifs $\ca{M}_\mathrm{coll}$ and
$\ca{M}_\mathrm{expa}$. For structural motif adjacency matrices, simply replace
$J_\mathrm{n}$, $J$ and $G$ with $J_0$, $J_\mathrm{s}$ and $G_\mathrm{s}$
respectively. Entry-wise products are denoted by $\circ$.

\vspace*{0.2cm}
\begin{table}[H]

\centering
\renewcommand{\arraystretch}{1.8}
\tiny

\begin{tabular}{ |c|c|c|c| }

\hline

Motif & $C$ & $C'$ & $M^\mathrm{func}$ \\

\hline

$\ca{M}_\mathrm{s}$ & & & $G + G^\top$ \\

\hline

$\ca{M}_\mathrm{d}$ & & & $\frac{1}{2} G_\mathrm{d}$ \\

\hline

$\ca{M}_1$ & $J^\top \circ (J G) + J^\top \circ (G J) + G^\top \circ (J J)$
& & $\frac{1}{3} \big(C + C^\top\big)$ \\

\hline

$\ca{M}_2$ & \rule{0pt}{2.7em}$\displaystyle
\begin{aligned}
& J^\top \circ (J_\mathrm{d} G) + J^\top \circ (G_\mathrm{d} J) + G^\top
\circ (J_\mathrm{d} J) \\
& + J^\top \circ (J G_\mathrm{d}) + J^\top \circ (G J_\mathrm{d}) +
G^\top \circ (J J_\mathrm{d}) \\
& + J_\mathrm{d} \circ (J G) + J_\mathrm{d} \circ (G J) + G_\mathrm{d}
\circ (J J)
\end{aligned}
$\rule[-2em]{0pt}{1em} & & $\frac{1}{4} \big(C + C^\top\big)$ \\

\hline

$\ca{M}_3$ & \rule{0pt}{2.7em}$\displaystyle
\begin{aligned}
& J \circ (J_\mathrm{d} G_\mathrm{d}) + J \circ (G_\mathrm{d}
J_\mathrm{d}) + G \circ (J_\mathrm{d} J_\mathrm{d}) \\
& + J_\mathrm{d} \circ (J_\mathrm{d} G) + J_\mathrm{d} \circ
(G_\mathrm{d} J) + G_\mathrm{d} \circ (J_\mathrm{d} J) \\
& + J_\mathrm{d} \circ (J G_\mathrm{d}) + J_\mathrm{d} \circ (G
J_\mathrm{d}) + G_\mathrm{d} \circ (J J_\mathrm{d})
\end{aligned}
$\rule[-2em]{0pt}{1em} & & $\frac{1}{5} \big(C + C^\top\big)$ \\

\hline

$\ca{M}_4$ & $ J_\mathrm{d} \circ (J_\mathrm{d} G_\mathrm{d}) +
J_\mathrm{d} \circ (G_\mathrm{d} J_\mathrm{d}) + G_\mathrm{d} \circ
(J_\mathrm{d} J_\mathrm{d}) $ & & $ \frac{1}{6} C$ \\

\hline

$\ca{M}_5$ & \rule{0pt}{2.7em}$\displaystyle
\begin{aligned}
& J \circ (J G) + J \circ (G J) + G \circ (J J) \\
& + J \circ (J G^\top) + J \circ (G J^\top) + G \circ (J J^\top) \\
& + J \circ (J^\top G) + J \circ (G^\top J) + G \circ (J^\top J)
\end{aligned}
$\rule[-2em]{0pt}{1em} & & $\frac{1}{3} \big(C + C^\top\big)$ \\

\hline

$\ca{M}_6$ & $J \circ (J G_\mathrm{d}) + J \circ (G J_\mathrm{d}) + G \circ
(J J_\mathrm{d}) + J_\mathrm{d} \circ (J^\top G)$ & $G_\mathrm{d} \circ
(J^\top J)$ & $\frac{1}{4} \big(C + C^\top + C' \big)$ \\

\hline

$\ca{M}_7$ & $J \circ (J_\mathrm{d} G) + J \circ (G_\mathrm{d} J) + G \circ
(J_\mathrm{d} J)$ & $J_\mathrm{d} \circ (J G^\top) + J_\mathrm{d} \circ (G
J^\top) + G_\mathrm{d} \circ (J J^\top)$ & $ \frac{1}{4} \big(C + C^\top +
C' \big)$ \\

\hline

$\ca{M}_8$ & $J \circ (G J_\mathrm{n}) + G \circ (J J_\mathrm{n})$ &
$J_\mathrm{n} \circ (J^\top G) + J_\mathrm{n} \circ (G^\top J)$ &
$\frac{1}{2} \big(C + C^\top + C' \big)$ \\

\hline

$\ca{M}_9$ & \rule{0pt}{1.9em}$\displaystyle
\begin{aligned}
& J \circ (J_\mathrm{n} G^\top) + G \circ (J_\mathrm{n} J^\top) +
J_\mathrm{n} \circ (J G) \\
& + J_\mathrm{n} \circ (G J) + J \circ (G^\top J_\mathrm{n}) + G \circ
(J^\top J_\mathrm{n})
\end{aligned}
$\rule[-1.3em]{0pt}{1em} & & $\frac{1}{2} \big(C + C^\top\big)$ \\

\hline

$\ca{M}_{10}$ & $J \circ (J_\mathrm{n} G) + G \circ (J_\mathrm{n} J)$ &
$J_\mathrm{n} \circ (J G^\top) + J_\mathrm{n} \circ (G J^\top)$ &
$\frac{1}{2} \big(C + C^\top + C' \big)$ \\

\hline

$\ca{M}_{11}$ & \rule{0pt}{1.9em}$\displaystyle
\begin{aligned}
& J_\mathrm{d} \circ (G J_\mathrm{n}) + G_\mathrm{d} \circ (J
J_\mathrm{n}) + J_\mathrm{n} \circ (J_\mathrm{d} G) \\
&  + J_\mathrm{n} \circ (G_\mathrm{d} J) + J \circ (G_\mathrm{d}
J_\mathrm{n}) + G \circ (J_\mathrm{d} J_\mathrm{n})
\end{aligned}
$\rule[-1.3em]{0pt}{1em} & & $\frac{1}{3} \big(C + C^\top\big)$ \\

\hline

$\ca{M}_{12}$ & \rule{0pt}{1.9em}$\displaystyle
\begin{aligned}
& J_\mathrm{d} \circ (J_\mathrm{n} G) + G_\mathrm{d} \circ (J_\mathrm{n}
J) + J_\mathrm{n} \circ (J G_\mathrm{d}) \\
& + J_\mathrm{n} \circ (G J_\mathrm{d}) + J \circ (J_\mathrm{n}
G_\mathrm{d}) + G \circ (J_\mathrm{n} J_\mathrm{d})
\end{aligned}
$\rule[-1.3em]{0pt}{1em} & & $ \frac{1}{3} \big(C + C^\top\big)$ \\

\hline

$\ca{M}_{13}$ & $J_\mathrm{d} \circ (G_\mathrm{d} J_\mathrm{n}) +
G_\mathrm{d} \circ (J_\mathrm{d} J_\mathrm{n}) + J_\mathrm{n} \circ
(J_\mathrm{d} G_\mathrm{d})$ & & $\frac{1}{4} \big(C + C^\top \big)$ \\

\hline

$\ca{M}_\mathrm{coll}$ & $J_\mathrm{n} \circ (J G^\top)$ & & $\frac{1}{2}
\big( C + C^\top \big)$ \\

\hline

$\ca{M}_\mathrm{expa}$ & $J_\mathrm{n} \circ (J^\top G)$ & & $\frac{1}{2}
\big( C + C^\top \big)$ \\

\hline

\end{tabular}
\caption{Functional motif adjacency matrix formulae}
\label{tab:motif_adj_mat_table}
\end{table}
\clearpage{}
\clearpage{}
\chapter{Further Notes}

\section{Computation} \label{sec:notes_computation}

\subsection{Hardware and software} \label{sec:notes_hardware}

The hardware used for computation was an \emph{Intel Core i7-4790} CPU at
3.60\,GHz, with 32\,GB of RAM. The software used was R 3.5.1
\cite{r_rsoftware}, along with several R packages:
%
%
\begin{itemize}
\item \textbf{igraph} \cite{r_igraph} for plotting networks
\item \textbf{LICORS} \cite{r_LICORS} for an implementation of $k$-means++
\item \textbf{mclust} \cite{r_mclust} for an implementation of ARI
\item
\textbf{rnaturalearth} \cite{r_rnaturalearth} for world territory boundary
data
\item \textbf{RSpectra} \cite{r_RSpectra} for eigendecomposition of
sparse matrices
\item \textbf{USAboundaries} \cite{r_USAboundaries} for US
county and state boundary data
\end{itemize}

\subsection{Timings for MAM computations} \label{sec:notes_timing}

We record timings (in seconds) for the MAM formulae given in
Table~\ref{tab:motif_adj_mat_table}. We test on DSBMs
(Section~\ref{sec:motif_dsbms}) with $k=1$, and vary the graph size $n$ and
sparsity parameter $p$.

\vspace*{0.3cm}
\begin{table}[H]
\centering \renewcommand{\arraystretch}{1.5}
\setlength\tabcolsep{0.2em} \scriptsize
\begin{tabular}{|c|c|c|c|c|c|c|c|c|c|c|c|c|c|c|c|}
\hline
\cellcolor[HTML]{E9E9E9} \smash{\raisebox{0.7pt}{$p$}} &
\cellcolor[HTML]{E9E9E9} $\ca{M}_\mathrm{s}$ &
\cellcolor[HTML]{E9E9E9} $\ca{M}_\mathrm{d}$ &
\cellcolor[HTML]{E9E9E9} $\ca{M}_1$ &
\cellcolor[HTML]{E9E9E9} $\ca{M}_2$ &
\cellcolor[HTML]{E9E9E9} $\ca{M}_3$ &
\cellcolor[HTML]{E9E9E9} $\ca{M}_4$ &
\cellcolor[HTML]{E9E9E9} $\ca{M}_5$ &
\cellcolor[HTML]{E9E9E9} $\ca{M}_6$ &
\cellcolor[HTML]{E9E9E9} $\ca{M}_7$ &
\cellcolor[HTML]{E9E9E9} $\ca{M}_8$ &
\cellcolor[HTML]{E9E9E9} $\ca{M}_9$ &
\cellcolor[HTML]{E9E9E9} $\ca{M}_{10}$ &
\cellcolor[HTML]{E9E9E9} $\ca{M}_{11}$ &
\cellcolor[HTML]{E9E9E9} $\ca{M}_{12}$ &
\cellcolor[HTML]{E9E9E9} $\ca{M}_{13}$ \\
\hline \cellcolor[HTML]{E9E9E9}
0.0001 & 0.013 & 0.012 & 0.017 & 0.029 & 0.034 & 0.015 & 0.028 &
0.022 & 0.022 & 0.019 & 0.030 & 0.019 & 0.042 & 0.021 & 0.016 \\
\hline \cellcolor[HTML]{E9E9E9}
0.001 & 0.013 & 0.011 & 0.016 & 0.035 & 0.027 & 0.017 & 0.028 &0.024 &
0.023 & 0.026 & 0.027 & 0.018 & 0.021 & 0.022 & 0.016 \\
\hline \cellcolor[HTML]{E9E9E9}
0.01 & 0.013 & 0.012 & 0.024 & 0.028 & 0.028 & 0.016 & 0.028 & 0.022 &
0.032 & 0.021 & 0.026 & 0.020 & 0.023 & 0.023 & 0.017 \\
\hline \cellcolor[HTML]{E9E9E9}
0.1 & 0.014 & 0.019 & 0.019 & 0.031 & 0.029 & 0.019 & 0.033 & 0.025 &
0.032 & 0.023 & 0.028 & 0.023 & 0.026 & 0.025 & 0.019
\\ \hline
\end{tabular}
\caption{Timings for MAM computation with $n=100$}%
\label{tab:timing_n_100}%
\end{table}

\begin{table}[H]
\centering \renewcommand{\arraystretch}{1.5}
\setlength\tabcolsep{0.2em} \scriptsize
\begin{tabular}{
|c|c|c|c|c|c|c|c|c|c|c|c|c|c|c|c|c|c| } \hline
\cellcolor[HTML]{E9E9E9} \smash{\raisebox{0.7pt}{$p$}} &
\cellcolor[HTML]{E9E9E9} $\ca{M}_\mathrm{s}$ &
\cellcolor[HTML]{E9E9E9} $\ca{M}_\mathrm{d}$ &
\cellcolor[HTML]{E9E9E9} $\ca{M}_1$ &
\cellcolor[HTML]{E9E9E9} $\ca{M}_2$ &
\cellcolor[HTML]{E9E9E9} $\ca{M}_3$ &
\cellcolor[HTML]{E9E9E9} $\ca{M}_4$ &
\cellcolor[HTML]{E9E9E9} $\ca{M}_5$ &
\cellcolor[HTML]{E9E9E9} $\ca{M}_6$ &
\cellcolor[HTML]{E9E9E9} $\ca{M}_7$ &
\cellcolor[HTML]{E9E9E9} $\ca{M}_8$ &
\cellcolor[HTML]{E9E9E9} $\ca{M}_9$ &
\cellcolor[HTML]{E9E9E9} $\ca{M}_{10}$ &
\cellcolor[HTML]{E9E9E9} $\ca{M}_{11}$ &
\cellcolor[HTML]{E9E9E9} $\ca{M}_{12}$ &
\cellcolor[HTML]{E9E9E9} $\ca{M}_{13}$ \\
\hline \cellcolor[HTML]{E9E9E9}
0.0001 & 0.13 & 0.14 & 0.14 & 0.32 & 0.14 & 0.13 & 0.14 & 0.31 &
0.13 & 0.21 & 0.22 & 0.21 & 0.20 & 0.34 & 0.16 \\
\hline \cellcolor[HTML]{E9E9E9}
0.001 & 0.30 & 0.13 & 0.15 & 0.16 & 0.16 & 0.14 & 0.16 & 0.32 & 0.14 &
0.48 & 0.37 & 0.29 & 0.31 & 0.29 & 0.17 \\
\hline \cellcolor[HTML]{E9E9E9}
0.01 & 0.11 & 0.14 & 0.17 & 0.19 & 0.14 & 0.13 & 0.21 & 0.18 & 0.18 &
0.64 & 0.73 & 0.89 & 0.46 & 0.56 & 0.18 \\
\hline \cellcolor[HTML]{E9E9E9}
0.1 & 0.23 & 0.22 & 0.60 & 1.1 & 0.57 & 0.24 & 1.4 & 0.86 & 0.69 &
1.5 & 2.3 & 1.6 & 1.6 & 1.6 & 0.67
\\ \hline
\end{tabular}
\caption{Timings
for MAM computation with $n=1000$} \label{tab:timing_n_1000}
\end{table}

\begin{table}[H] \centering \renewcommand{\arraystretch}{1.5}
\setlength\tabcolsep{0.2em} \scriptsize
\begin{tabular}{
|c|c|c|c|c|c|c|c|c|c|c|c|c|c|c|c|c|c| } \hline
\cellcolor[HTML]{E9E9E9} \smash{\raisebox{0.7pt}{$p$}} &
\cellcolor[HTML]{E9E9E9} $\ca{M}_\mathrm{s}$ &
\cellcolor[HTML]{E9E9E9} $\ca{M}_\mathrm{d}$ &
\cellcolor[HTML]{E9E9E9} $\ca{M}_1$ &
\cellcolor[HTML]{E9E9E9} $\ca{M}_2$ &
\cellcolor[HTML]{E9E9E9} $\ca{M}_3$ &
\cellcolor[HTML]{E9E9E9} $\ca{M}_4$ &
\cellcolor[HTML]{E9E9E9} $\ca{M}_5$ &
\cellcolor[HTML]{E9E9E9} $\ca{M}_6$ &
\cellcolor[HTML]{E9E9E9} $\ca{M}_7$ &
\cellcolor[HTML]{E9E9E9} $\ca{M}_8$ &
\cellcolor[HTML]{E9E9E9} $\ca{M}_9$ &
\cellcolor[HTML]{E9E9E9} $\ca{M}_{10}$ &
\cellcolor[HTML]{E9E9E9} $\ca{M}_{11}$ &
\cellcolor[HTML]{E9E9E9} $\ca{M}_{12}$ &
\cellcolor[HTML]{E9E9E9} $\ca{M}_{13}$ \\
\hline \cellcolor[HTML]{E9E9E9}
0.0001 & 11 & 12 & 12 & 12 & 12 & 12 & 12 & 12 & 12 &
41 & 55 & 37 & 38 & 34 & 15 \\
\hline \cellcolor[HTML]{E9E9E9}
0.001 & 13 & 12 & 13 & 13 & 12 & 12 & 13 & 12 &
12 & 61 & 89 & 54 & 56 & 48 & 15 \\
\hline \cellcolor[HTML]{E9E9E9}
0.01 & 13 & 13 & 36 & 36 & 14 & 13 & 82 & 36 & 36 &
150 & 230 & 130 & 130 & 99 & 36 \\
\hline \cellcolor[HTML]{E9E9E9}
0.1 & 33 & 31 & 170 & 260 & 160 & 53 & 410 &
210 & 210 & 700 & 1100 & 520 & 760 & 580 & 150
\\ \hline
\end{tabular}
\caption{Timings
for MAM computation with $n=10 \, 000$} \label{tab:timing_n_10000}
\end{table}

\section{Data preprocessing} \label{sec:notes_preprocessing}

All real networks were preprocessed by restriction to their largest connected
component. The Unicode Languages network in
Section~\ref{sec:bipartite_languages} was also preprocessed to remove
territories with under one million inhabitants and languages with under one
million speakers. Vertex and edge counts of all networks are stated
\emph{after} this preprocessing.

\section{US map} \label{sec:notes_us_map}
%
\vspace*{-0.8cm}
\begin{figure}[H]
\centering
%\includegraphics[scale=0.6,draft=false]{%
%../../results/us_migration/us_migration_map_state_names.pdf}
\vspace*{-0.5cm} \caption{US map with state boundaries and state
abbreviations} \label{fig:notes_us_map}
\end{figure}

\section{Word count}

The word count of this dissertation is 6230
\unskip, obtained
using \TeX \hspace*{-0.15cm} count by running
%
\begin{center}
\texttt{texcount -relaxed -inc -0 -sum=1,1,1,0,0,0,0\,}.
\end{center}
%
%The final dissertation should be no longer than 7,500 words, this usually
%equates to 25--30 pages. The word count may exclude any table of contents,
%all mathematical equations and symbols, diagrams, tables, bibliography and
%the texts of computer programs. However any preface, footnotes, and
%appendices must be included

\clearpage{}
%TC:ignore

% add the bibliography to the contents page
\pagestyle{empty}
\cleardoublepage
\phantomsection
\addcontentsline{toc}{chapter}{References}
% change bibliography name to References
\renewcommand{\bibname}{References}
\pagestyle{fancy}
\fancyhead[RO]{\itshape{\nouppercase{References}}}
\bibliography{refs}
\bibliographystyle{abbrv}

%TC:endignore
\end{document}
